\documentclass{article}
\title{Reporte de Actividades - Agosto 2012 - Julio 2013}
\author{Jaime E. Forero Romero\\Profesor Asistente - Departamento de
  F\'isica\\Universidad de los Andes}  
\begin{document}

\maketitle

\section*{Docencia}

\begin{tabular}{p{7.0cm} p{1.5cm}}\hline
Curso & Semestre\\\hline
F\'isica 1 & 2012-2 \\
M\'odulo Introducci\'on a la F\'isica & 2012-2 \\
F\'isica 1 & 2013-1\\
F\'isica Computacional & 2013-1\\
Seminario de pregrado de Astronom\'ia & 2013-1\\\hline
\end{tabular}

\section*{Presentaciones cient\'ificas}

\begin{tabular}{lp{2.0cm} p{1.2cm} p{1.5cm} p{2cm} p{5cm}}\hline
N$^{o}$ & Fecha & Pa\'is & Ciudad & Lugar & T\'itulo \\\hline
8 & 25-06-2013 & Korea & Seoul & Korea Institute for Advanced Study &
The Local Group in the Cosmic Web\\ 
7 & 19-06-2013 & Korea & Daejeon & Korea Astronomy and Space Science
Institute & Professional and Social Astronomy in Colombia\\ 
6 & 9-05-2013 & Spain & Madrid & CLUES meeting & The place of the
Local Group in the Cosmic Web\\ 
5 & 26-04-2013 & Colombia & Tunja & Universidad Pedag\'ogica y
Tecnol\'ogica de Colombia & Desde la V\'ia L\'actea hasta las Galaxias
m\'as distantes\\ 
4 &17-01-2013 & Chile & Santiago  & Universidad Andr\'es Bello &
Stochasticity in High Redshift Galaxies\\ 
3 &14-01-2013 & Chile & Santiago  & Universidad de Chile&
Stochasticity in High Redshift Galaxies\\
2 & 11-01-2013 & Chile & Santiago  & Pontificia Universidad Cat\'olica&
Stochasticity in High Redshift Galaxies\\
1 & 17-12-2012 & USA & Santa Cruz, CA & University of California Santa
Cruz & Recent advances on the cosmic web description\\  \hline
\end{tabular}

\section*{Presentaciones p\'ublicas}

\begin{tabular}{lp{2.0cm} p{1.2cm} p{1.5cm} p{2cm} p{5cm}}\hline
N$^{o}$ & Fecha & Pa\'is & Ciudad & Lugar & T\'itulo\\\hline
3 & 26-04-2013 & Colombia & Tunja & Casa del Banco de la Rep\'ublica &
Formaci\'on de galaxias: una novela de 14 mil millones de a\~nos\\  
2 & 27-03-2013 & Colombia & Bogot\'a & Planetario de Bogot\'a & Un mapa del
Universo\\
1 & 12-10-2012 & Colombia & Bogot\'a & Universidad de los Andes & Pecha
Kucha Vol. 9.\\ \hline
\end{tabular}

\section*{Investigaci\'on}

\subsection*{Art\'iculos publicados}

\begin{itemize}
\item {\it The MultiDark Database: Release of the Bolshoi and MultiDark Cosmological Simulations} , K. Riebe , A. M. Partl, H. Enke, {\bf J.E. Forero-Romero}, S. Gottloeber, A. Klypin, G. Lemson, F. Prada, J. R. Primack, M. Steinmetz, V. Turchaninov, Astronomische Nachrichten, 334, 691, 2013.

\item {\it The kinematics of the Local Group in a cosmological context}, 
{\bf J.E. Forero-Romero}, Y. Hoffman, S. Bustamante, S. Gottloeber, G. Yepes, ApJ Letters, 767, 1, 2013

\item {\it The velocity shear tensor: tracer of halo alignment}, Libeskind N., Hoffman Y., {\bf Forero-Romero} J.E., Gottloeber S., Knebe A., Steinmentz M., Klypin A., MNRAS 428, 2489, 2013

\item {\it A kinematic classification of the cosmic web}, Y. Hoffman, O. Metuki , G. Yepes, S. Gottloeber, {\bf J. E. Forero-Romero}, N. I. Libeskind, A. Knebe, MNRAS, 425, 2049, 2012
\end{itemize}

\subsection*{Art\'itculos enviados}

\begin{itemize}
\item {\it Mass and occupation fraction of dark matter halos hosting Lyman-$\alpha$
  emitters at $z\sim 3$}, {\bf Forero-Romero} J.E., Mej\'ia-Restrepo
  J.E., MNRAS submitted 2013.
\item {\it X-ray absorption lines atlas for active galactic nuclei
  warm absorbers I: The case of NGC 3783}, Ram\'irez  J.M, Palmeri P.,
  Garc\'ia J., {\bf Forero-Romero} J.E., ApJ submitted 2013.
\end{itemize}

\subsection*{Cap\'itulos de libro}
\begin{itemize}
\item {\it Computer Simulations in a Cosmological Context.} Ayala L, {\bf Forero-Romero} J.E. Computer Simulations and the Changing Face of Scientific Experimentation (ISBN 1-4438-4792-5) pp. 172-192. Cambridge Scholars Publishing, 2013.
\end{itemize}

\section*{Desarrollo Institucional}

\subsection*{Comit\'es}

\begin{itemize}
\item {Miembro del comit\'e de examen de conocimientos}: 2013-01
\end{itemize}

\subsection*{Visibilidad internacional}

\begin{itemize}
\item {Organizador de la conferencia internacional "Astronom\'ia en los
  Andes" realizada en Julio del 2013 en las instalaciones de la Universidad. 5KEuro de apoyo del ICTP para organizaci\'on.}
\item {Visita a Corea del Sur para iniciar procesos de cooperaci\'on
  cient\'ifica en astronom\'ia. 3KUSD de apoyo de KIAS para la visita.}
\end{itemize}

\subsection*{Popularizaci\'on de la ciencia}
\begin{itemize}
\item {Miembro del comit\'e cient\'ifico del planetario de Bogot\'a.}
\item {Co-director del proyecto Astronom\'ia Perif\'erica apoyado por
  la Uni\'on Astron\'omica Internacional. 4KEuro de apoyo de la Uni\'on Astron\'omica Internacional para su realizaci\'on.}
\end{itemize}


\section*{Objetivos Agosto 2013 - Julio 2014}
\subsection*{Investigaci\'on}
\subsection*{Docencia}
\subsection*{Desarrollo Institucional}

\end{document}
