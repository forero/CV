\documentclass{article}
\title{Reporte de Actividades - Agosto 2012 - Julio 2013}
\author{Jaime E. Forero Romero\\Profesor Asistente - Departamento de
  F\'isica\\Universidad de los Andes}  
\begin{document}

\maketitle

\section*{Docencia}

\begin{tabular}{p{6.0cm} p{6.0cm}}\hline
Curso & Semestre\\\hline
F\'isica II & 2014-1. $60$ inscritos. Calificaci\'on Estudiantes: 3.55/4.0\\
M\'etodos computacionales & 2014-1. $12$ inscritos. Calificaci\'on Estudiantes: 3.76/4.0\\
Seminario de pregrado de Astronom\'ia & 2014-1. $\sim 10$ inscritos.\\\hline
F\'isica II & 2013-2. $61$ inscritos. Calificaci\'on Estudiantes 3.47/4.0\\
F\'isica Computacional & 2013-2. $28$ inscritos. Calificac\'on Estudiantes: 3.36/4.0\\
Seminario de pregrado de Astronom\'ia & 2013-2. $\sim 10$ inscritos.\\\hline
\end{tabular}

\section*{Presentaciones cient\'ificas}

\begin{tabular}{lp{2.0cm} p{1.2cm} p{1.8cm} p{2cm} p{5cm}}\hline
N$^{o}$ & Fecha & Pa\'is & Ciudad & Lugar & T\'itulo \\\hline
5 & 5-12-2013 & Chile & Santiago & Pontificia Universidad Cat\'olica & The Local Group in an Explicit Cosmological Context\\\hline
4 & 27-11-2013 & Brazil & Florian\'opolis & Latin American IAU Regional Meeting & The Local Group in an Explicit Cosmological Context\\\hline
3 & 21-10-2013 & Colombia & Bogot\'a & Coloquio del Departamento de F\'isica Universidad Nacional & Astronom\'ia y f\'isica fundamental: conexiones a trav\'es de la formaci\'on de galaxias\\\hline
2 & 2-10-2013 & Colombia & Bogot\'a & I Foro Internacional de Ciencias - Uniandes & Astronom\'ia y f\'isica fundamental: conexiones a trav\'es de la formaci\'on de galaxias\\\hline
1 & 26-8-2013 & Colombia & Armenia & Congreso Colombiano de F\'isica & Astronom\'ia y F\'isica fundamental: conexiones a trav\'es de la formaci\'on de galaxias.\\  \hline
\end{tabular}

\section*{Presentaciones p\'ublicas}

\begin{tabular}{lp{2.0cm} p{1.2cm} p{1.5cm} p{2cm} p{5cm}}\hline
N$^{o}$ & Fecha & Pa\'is & Ciudad & Lugar & T\'itulo\\\hline
2 & 29-10-2013 & Colombia & Bogot\'a & Planetario de Bogot\'a & El lado oscuro del Universo\\ \hline
1 & -& Colombia & Bogot\'a & Planetario de Bogot\'a & Fluid Skies: Arte y Astronom\'ia\\ \hline
\end{tabular}

\section*{Investigaci\'on}

\subsection*{Art\'iculos publicados}

\begin{enumerate}
\item 
\end{enumerate}

\section*{Desarrollo Institucional}

\subsection*{Comit\'es}

\begin{itemize}
\item {Miembro del comit\'e de examen de conocimientos}: 2013-2
\end{itemize}

\subsection*{Visibilidad internacional}

\begin{itemize}
\item 
\end{itemize}

\subsection*{Popularizaci\'on de la ciencia}
\begin{itemize}
\item {Miembro del comit\'e cient\'ifico del planetario de Bogot\'a.}
\item {Co-director del proyecto Astronom\'ia Perif\'erica apoyado por
  la Uni\'on Astron\'omica Internacional. 4KEuro de apoyo de la Uni\'on Astron\'omica Internacional para su realizaci\'on.} 
\end{itemize}


\section*{Funding externo}
\begin{tabular}{lp{1.8cm} p{2.0cm} p{3.5cm} p{2.3cm} p{2.3cm}}\hline
N$^{o}$ & Fecha & Duraci\'on & Instituci\'on & Proyecto & Monto (USD)\\\hline
1 & 26.12.2013 & 3 semanas & Korean Institute for Advanced Study & Hospedaje y vi\'aticos & 1260\\\hline
\end{tabular}

\section*{Objetivos Agosto 2014 - Julio 2015}
\subsection*{Investigaci\'on}

\subsection*{Docencia}

\subsection*{Desarrollo Institucional}



\end{document}
