\documentclass{article}
\addtolength{\oddsidemargin}{-.875in}
\addtolength{\evensidemargin}{-.875in}
\addtolength{\textwidth}{1.0in}
\addtolength{\topmargin}{-0.5in}
\addtolength{\textheight}{1.00in}
\usepackage{wrapfig}
\usepackage{sidecap}
\usepackage[pdftex]{graphicx}
\usepackage[utf8]{inputenc}
\usepackage[spanish]{babel}

\title{Reporte de Actividades - Agosto 2013 - Julio 2014}
\author{Jaime E. Forero Romero\\Profesor Asistente - Departamento de
  F\'isica\\Universidad de los Andes}  
\begin{document}

\maketitle
\tableofcontents
\newpage

\section{Docencia}

\subsection{Cursos dictados}
\begin{tabular}{p{6.0cm} l c p{2cm}}\hline
Curso & Semestre & Inscritos & Calificaci\'on Estudiantes\\\hline
F\'isica II & 2014-10 & $60$  & 3.55/4.0\\
M\'etodos computacionales & 2014-10 & $12$ & 3.76/4.0\\
Seminario de pregrado de Astronom\'ia & 2014-10 & 10 & \\\hline
F\'isica II & 2013-20 & $61$ & 3.47/4.0\\
F\'isica Computacional & 2013-20 & $28$ & 3.36/4.0\\
Seminario de pregrado de Astronom\'ia & 2013-20 & 10 & \\\hline
\end{tabular}

\subsection{Desarrollo de nuevos cursos}



\newpage
\section{Investigaci\'on}

\subsection{Art\'iculos publicados}

\begin{enumerate}
\item 
\end{enumerate}

\subsection{Direcci\'on de estudiantes}

\begin{enumerate}
\item{Felipe Leonardo Gomez Cortes. Primer a\~no de doctorado.}

\item{Juan Nicolas Garavito Camargo. Primer a\~no de maestr\'ia.}
\end{enumerate}

\subsection{Funding externo}
\begin{tabular}{l l l p{3.5cm} p{2.3cm} c}\hline
N$^{o}$ & Fecha & Duraci\'on & Instituci\'on & Proyecto & Monto (USD)\\\hline
3 & 1.06.2014 & 1 semana & IAU grant para asistencia a un simposio en
Estonia & Vuelo y hospedaje & 3000\\\hline
2 & 1.05.2014 & 4 semanas & Heidelberg Institute for Theoretical
Studies / (DAAD grant) & Vuelo, hospedaje y vi\'aticos & 5000\\\hline
1 & 26.12.2013 & 3 semanas & Korean Institute for Advanced Study &
Vuelo, hospedaje y vi\'aticos & 3000\\\hline
\end{tabular}

\subsection{Presentaciones t\'ecnicas}

\begin{tabular}{l c l l p{2cm} p{5cm}}\hline
N$^{o}$ & Fecha & Pa\'is & Ciudad & Lugar & T\'itulo \\\hline
9 & 26-08-2014 & Germany & Potsdam & 11th Potsdam Thinkshop & The
place of the Local Group in the Cosmic Web\\ \hline
8 & 24-06-2014 & Estonia & Tallinn & IAU symposium 308 & The place of the
Local Group in the cosmic web\\ \hline
7 & 19-05-2014 & Colombia & Bogot\'a & Universidad de los Andes &
Astrophysical evidence for the existence of Dark Matter\\ \hline
6 & 8-05-2014 & Colombia & Bogot\'a & Universidad de los Andes &
Alcock-Paczinsky test on the galaxy gradient field \\ \hline
5 & 5-12-2013 & Chile & Santiago & Pontificia Universidad Cat\'olica &
The Local Group in an Explicit Cosmological Context\\\hline 
4 & 27-11-2013 & Brazil & Florian\'opolis & Latin American IAU
Regional Meeting & The Local Group in an Explicit Cosmological
Context\\\hline 
3 & 21-10-2013 & Colombia & Bogot\'a & Coloquio del Departamento de
F\'isica Universidad Nacional & Astronom\'ia y f\'isica fundamental:
conexiones a trav\'es de la formaci\'on de galaxias\\\hline 
2 & 2-10-2013 & Colombia & Bogot\'a & I Foro Internacional de Ciencias
- Uniandes & Astronom\'ia y f\'isica fundamental: conexiones a
trav\'es de la formaci\'on de galaxias\\\hline 
1 & 26-8-2013 & Colombia & Armenia & Congreso Colombiano de F\'isica &
Astronom\'ia y F\'isica fundamental: conexiones a trav\'es de la
formaci\'on de galaxias.\\  \hline 
\end{tabular}
\section{Desarrollo Institucional}

\subsection{Comit\'es}
\begin{itemize}
\item {Representante del Departamento de F\'isica en el comit\'e
  ad-hoc c\'omputo de alto rendimiento de la facultad de ciencias}:
  2014-10  
\item {Miembro del comit\'e de examen de conocimientos}: 2013-20 
\end{itemize}


\subsection{Visibilidad internacional}
\begin{itemize}
\item {Lidero una propuesta para la creaci\'on de una Oficina Regional
  de Astronom\'ia para el Desarrollo. Esta Oficina es una red
  colaboraci\'on entre Colombia, Venezuela, Ecuador, Per\'u y Chile
  con el patrocinio de la Uni\'on Astron\'omica Internacional.}  
\end{itemize}

\subsection{Popularizaci\'on de la ciencia}
\begin{itemize}
\item {He dado charlas de divulgaci\'on para el p\'ublico general en
  el Planetario de Bogot\'a y diferentes eventos en el pa\'is.} 
\item {Doy apoyo al seminario de astronom\'ia del Planetario de
  Bogot\'a. Un espacio para que estudiantes de astronom\'ia de la
  ciudad presenten sus investigaciones.}  
\end{itemize}

\subsection{Presentaciones p\'ublicas}

\begin{tabular}{l l l l p{3cm} p{4cm}}\hline
N$^{o}$ & Fecha & Pa\'is & Ciudad & Lugar & T\'itulo\\\hline
6 & 20-09-2014 & Colombia & Bogot\'a & Seminario de Estudiantes & Una
carrera astron\'omica \\ \hline
5 & 6-05-2014 & Colombia & Manizales & Festival Internacional de la
Imagen & Interacciones entre arte y ciencia\\ \hline
4 & 15-03-2014  & Colombia & Villa de Leyva & Encuentro de
diso\~nadores & Arte, Astronomía y Ritual \\ \hline
3 & 29-10-2013 & Colombia & Bogot\'a & Planetario de Bogot\'a & El
lado oscuro del Universo\\ \hline 
2 & 10-08-2013 & Colombia & Bogot\'a & Curso de Fundamentos de
Astronom\'ia y Astrof\'isica - Uniandes & Galaxias y
Cosmolog\'ia\\    \hline
1 & 29-10-2013 & Colombia & Bogot\'a & Planetario de Bogot\'a & El
lado oscuro del Universo\\ \hline 
\end{tabular}


\newpage
\section{Objetivos Agosto 2014 - Julio 2015}
\subsection{Investigaci\'on}

\subsection{Docencia}

\subsection{Desarrollo Institucional}



\end{document}
