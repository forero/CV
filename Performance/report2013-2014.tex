\documentclass{article}
\addtolength{\oddsidemargin}{-.875in}
\addtolength{\evensidemargin}{-.875in}
\addtolength{\textwidth}{1.0in}
\addtolength{\topmargin}{-0.5in}
\addtolength{\textheight}{1.00in}
\usepackage{wrapfig}
\usepackage{sidecap}
\usepackage[pdftex]{graphicx}
\usepackage[utf8]{inputenc}
\usepackage[spanish]{babel}

\title{Reporte de Actividades - Agosto 2013 - Julio 2014}
\author{Jaime E. Forero Romero\\Profesor Asistente - Departamento de
  F\'isica\\Universidad de los Andes}  
\begin{document}

\maketitle
\tableofcontents
\newpage

\section{Docencia}

\subsection{Cursos dictados}
\begin{tabular}{p{6.0cm} l c p{2cm}}\hline
Curso & Semestre & Inscritos & Calificaci\'on Estudiantes\\\hline
F\'isica II & 2014-10 & $60$  & 3.55/4.0\\
M\'etodos computacionales & 2014-10 & $12$ & 3.76/4.0\\
Seminario de pregrado de Astronom\'ia & 2014-10 & 10 & \\\hline
F\'isica II & 2013-20 & $61$ & 3.47/4.0\\
F\'isica Computacional & 2013-20 & $28$ & 3.36/4.0\\
Seminario de pregrado de Astronom\'ia & 2013-20 & 10 & \\\hline
\end{tabular}

\subsection{Desarrollo de nuevos cursos}
Desarroll\'e los programas de los cursos \emph{Herramientas
  Computacionales} y \emph{M\'etodos Computacionales} del nuevo
p\'ensum de la carrera de F\'isica.



\section{Investigaci\'on}

\subsection{Art\'iculos publicados}
Estudiantes de posgrado de Uniandes est\'an subrayados.


\begin{enumerate}
\item[5.]{\it The impact of gas bulk rotation on the lyman-alpha line}
  \underline{J.N. Garavito-Camargo},  {\bf J.E. Forero-Romero}, M. Dijkstra,
  Astrophysical Journal
  accepted, 2014. 

\item[4.]{\it } {\it Target allocation yields for massively
  multiplexed spectroscopic surveys with fibers} W. Saunders,
  S. Smedley, P. Guillingham, {\bf J.E. Forero-Romero}, S. Jouvel,
  B. Nord, Proceedings SPIE 9150, Modeling, Systems Engineering, and
  Project Management for Astronomy VI, 2014.

\item[3.]{\it Systematic uncertainties from halo asphericity in dark
  matter searches}  N. Bernal, {\bf J.E. Forero-Romero}, R. Garani,
  S. Palomares-Ruiz, Journal of Cosmology and Astroparticle Physics,
  09, 004, 2014. 

\item[2.]{\it Cosmic web alignments with the shape, angular momentum
  and peculiar velocities of dark matter halos}, 
  {\bf J.E. Forero-Romero}, S. Contreras, N. Padilla, Monthly Notices
  of the Royal Astronomical Society, 443,
  1090, 2014. 

\item[1.]{\it The abundance of Bullet Groups in $\Lambda$CDM},
  J. G. Fern\'andez-Trincado, {\bf J. E. Forero-Romero}, G. Foex,
  T. Verdugo, V. Motta, Astrophysical Journal Letters, 787, L32, 2014.
\end{enumerate}

\subsection{Direcci\'on de estudiantes}

\begin{enumerate}
\item{Felipe Leonardo G\'omez Cortes. Primer a\~no de doctorado. Ganador
  de concurso de COLCIENCIAS para hacer una pasant\'ia de un semestre en
  Purdue. Primer art\'iculo en preparaci\'on.} 

\item{Juan Nicol\'as Garavito Camargo. Primer a\~no de
  maestr\'ia. Seleccionado en el verano del 2014 para la escuela de
  astronom\'ia del Vaticano. Solamente 20 estudiantes de todo el mundo
asisten anualmente. Su primer art\'iculo ya ha sido publicado, hay un segundo
art\'iculo en preparaci\'on.}
\end{enumerate}

\subsection{Funding externo}
\begin{tabular}{l l l p{3.5cm} p{2.3cm} c}\hline
N$^{o}$ & Fecha & Duraci\'on & Instituci\'on & Proyecto & Monto (USD)\\\hline
3 & 1.06.2014 & 1 semana & IAU grant para asistencia a un simposio en
Estonia & Vuelo y hospedaje & 3000\\\hline
2 & 1.05.2014 & 4 semanas & Heidelberg Institute for Theoretical
Studies / (DAAD grant) & Vuelo, hospedaje y vi\'aticos & 5000\\\hline
1 & 26.12.2013 & 3 semanas & Korean Institute for Advanced Study &
Vuelo, hospedaje y vi\'aticos & 3000\\\hline
\end{tabular}

\subsection{Presentaciones t\'ecnicas}

\begin{tabular}{l c l l p{2cm} p{5cm}}\hline
N$^{o}$ & Fecha & Pa\'is & Ciudad & Lugar & T\'itulo \\\hline
8 & 24-06-2014 & Estonia & Tallinn & IAU symposium 308 & The place of the
Local Group in the cosmic web\\ \hline
7 & 19-05-2014 & Colombia & Bogot\'a & Universidad de los Andes &
Astrophysical evidence for the existence of Dark Matter\\ \hline
6 & 8-05-2014 & Colombia & Bogot\'a & Universidad de los Andes &
Alcock-Paczinsky test on the galaxy gradient field \\ \hline
5 & 5-12-2013 & Chile & Santiago & Pontificia Universidad Cat\'olica &
The Local Group in an Explicit Cosmological Context\\\hline 
4 & 27-11-2013 & Brazil & Florian\'opolis & Latin American IAU
Regional Meeting & The Local Group in an Explicit Cosmological
Context\\\hline 
3 & 21-10-2013 & Colombia & Bogot\'a & Coloquio del Departamento de
F\'isica Universidad Nacional & Astronom\'ia y f\'isica fundamental:
conexiones a trav\'es de la formaci\'on de galaxias\\\hline 
2 & 2-10-2013 & Colombia & Bogot\'a & I Foro Internacional de Ciencias
- Uniandes & Astronom\'ia y f\'isica fundamental: conexiones a
trav\'es de la formaci\'on de galaxias\\\hline 
1 & 26-8-2013 & Colombia & Armenia & Congreso Colombiano de F\'isica &
Astronom\'ia y F\'isica fundamental: conexiones a trav\'es de la
formaci\'on de galaxias.\\  \hline 
\end{tabular}
\section{Desarrollo Institucional}

\subsection{Comit\'es}
\begin{itemize}
\item {Representante del Departamento de F\'isica en el comit\'e
  ad-hoc c\'omputo de alto rendimiento de la facultad de ciencias}:
  2014-10  
\item {Miembro del comit\'e de examen de conocimientos}: 2013-20 
\end{itemize}


\subsection{Visibilidad nacional e internacional}
\begin{itemize}
\item {Doy apoyo al seminario de astronom\'ia del Planetario de
  Bogot\'a. Un espacio para que estudiantes de astronom\'ia de la
  ciudad presenten sus investigaciones.}  
\item {Lidero una propuesta para la creaci\'on de una Oficina Regional
  de Astronom\'ia para el Desarrollo. Esta Oficina es una red
  colaboraci\'on entre Colombia, Venezuela, Ecuador, Per\'u y Chile
  con el patrocinio de la Uni\'on Astron\'omica Internacional.}  
\end{itemize}

\subsection{Presentaciones p\'ublicas}

\begin{tabular}{l l l l p{3cm} p{4cm}}\hline
N$^{o}$ & Fecha & Pa\'is & Ciudad & Lugar & T\'itulo\\\hline
5 & 6-05-2014 & Colombia & Manizales & Festival Internacional de la
Imagen & Interacciones entre arte y ciencia\\ \hline
4 & 15-03-2014  & Colombia & Villa de Leyva & Encuentro de
diso\~nadores & Arte, Astronomía y Ritual \\ \hline
3 & 29-10-2013 & Colombia & Bogot\'a & Planetario de Bogot\'a & El
lado oscuro del Universo\\ \hline 
2 & 10-08-2013 & Colombia & Bogot\'a & Curso de Fundamentos de
Astronom\'ia y Astrof\'isica - Uniandes & Galaxias y
Cosmolog\'ia\\    \hline
1 & 29-10-2013 & Colombia & Bogot\'a & Planetario de Bogot\'a & El
lado oscuro del Universo\\ \hline 
\end{tabular}



\section{Objetivos Agosto 2014 - Julio 2015}


\subsection{Docencia}
Construcci\'on de una propuesta para la creaci\'on de un nuevo curso
CBU. 

Organizaci\'on de un curso de verano (2015) en temas de cosmolog\'ia
computacional y observacional.


\subsection{Investigaci\'on}
3 publicaciones internacionales. Al menos una de ellas con un
estudiante de posgrado de Uniandes.

Dar una charla internacional sobre mi investigaci\'on.

Direcci\'on de una tesis de pregrado.

\subsection{Desarrollo Institucional}
Participaci\'on en el comit\'e  de c\'omputo de alto rendimiento de la
Facultad de Ciencias.



\end{document}
