%%%%%%%%%%%%%%%%%%%%%%%%%%%%%%%%%%%%%%%%%
% Plasmati Graduate CV
% LaTeX Template
% Version 1.0 (24/3/13)
%
% This template has been downloaded from:
% http://www.LaTeXTemplates.com
%
% Original author:
% Alessandro Plasmati (alessandro.plasmati@gmail.com)
%
% License:
% CC BY-NC-SA 3.0 (http://creativecommons.org/licenses/by-nc-sa/3.0/)
%
%%%%%%%%%%%%%%%%%%%%%%%%%%%%%%%%%%%%%%%%%

%----------------------------------------------------------------------------------------
%	PACKAGES AND OTHER DOCUMENT CONFIGURATIONS
%----------------------------------------------------------------------------------------

\documentclass[a4paper,10pt]{article} % Default font size and paper size

\usepackage{fontspec} % For loading fonts
\defaultfontfeatures{Mapping=tex-text}
%\setmainfont[SmallCapsFont = Fontin SmallCaps]{Fontin} % Main document font

\usepackage{xltxtra,url,parskip} % Formatting packages

\usepackage[usenames,dvipsnames]{xcolor} % Required for specifying custom colors

\usepackage[big]{layaureo} % Margin formatting of the A4 page, an alternative to layaureo can be \usepackage{fullpage}
% To reduce the height of the top margin uncomment: \addtolength{\voffset}{-1.3cm}

\usepackage{hyperref} % Required for adding links	and customizing them
\definecolor{linkcolour}{rgb}{0,0.2,0.6} % Link color
\hypersetup{colorlinks,breaklinks,urlcolor=linkcolour,linkcolor=linkcolour} % Set link colors throughout the document

\usepackage{titlesec} % Used to customize the \section command
\titleformat{\section}{\Large\scshape\raggedright}{}{0em}{}[\color{black} \titlerule] % Text formatting of sections

\titleformat{\subsection}{\large\scshape\raggedright}{}{0em}{}[\color{black} \titlerule] % Text formatting of sections

\titleformat{\subsubsection}{\scshape\raggedright}{}{0em}{} % Text formatting of sections


\titlespacing{\section}{0pt}{4pt}{4pt} % Spacing around sections

\usepackage{fancyhdr}
\usepackage{lastpage}
 
\pagestyle{fancy}
\fancyhf{}
 
\rfoot{Page \thepage \hspace{1pt} of \pageref{LastPage}}

\begin{document}

\pagestyle{empty} % Removes page numbering



%----------------------------------------------------------------------------------------
%	PERSONAL INFORMATION
%----------------------------------------------------------------------------------------

\par{\centering{\Huge \color{black}  Jaime Ernesto Forero Romero} \bigskip\par} % Your name


\begin{tabular}{rl}
\textsc{Fecha y lugar de nacimiento:} & Octubre 14, 1981. Bogotá, Colombia.\\
\textsc{Sitio Web:} & \url{http://wwwprof.uniandes.edu.co/~je.forero}\\
\textsc{Correo Electrónico:} & \href{mailto:je.forero@uniandes.edu.co}{je.forero@uniandes.edu.co}\\
\end{tabular}


\color{red}
\section{Perfil y apuesta política}
\color{black}

Experto en cosmolog\'ia, astrof\'isica e inteligencia artificial. 
Profesor universitario. Participante en diversos colectivos sociales en la  zona Centro-Oriente en Bogotá.
Le apuesto a la ciencia  abierta y consciente para el beneficio de los sectores populares;
a los datos públicos y al código abierto para facilitar la participación popular, y a la
inteligencia artificial y analítica de datos para tomar mejores decisiones.
Trabajo para Cultivar Conciencias Cosmológicas Colectivas.
\color{red}
\subsection{Educación}
\color{black}


\begin{tabular}{rl}	
\textsc{Noviembre 2007} & {\bf PhD en Física}\\
\textsc{Agosto 2005}& de la \'Ecole Normale Sup\'erieure de Lyon (Francia) \\
&\\
\textsc{Julio 2005} & {\bf MSc en Física (Magist\`ere Interuniversitaire)}\\  
\textsc{Agosto 2003}& de la \'Ecole Normale Sup\'erieure (Paris, Francia) \\
&\\
\textsc{Julio 2003} & {\bf Estudios de pregrado en Física} (3$^{er}$ \& 4$^{o}$ año)\\
\textsc{Agosto 2001} & en el Instituto Balseiro (Bariloche, Argentina)\\
&\\
\textsc{Julio 2001}  & {\bf Estudios de pregrado en Física} (1$^{er}$ \& 2$^{o}$ año)\\
\textsc{Julio 1999}& un la Universidad Nacional de Colombia (Bogotá)\\
\end{tabular}

\color{red}
\subsection{Experiencia laboral}
\color{black}

\begin{tabular}{rl}	
 \textsc{Actualidad} &  {\bf Profesor de Planta}\\
 \textsc{Agosto 2012} & Departamento de Física, Universidad de los Andes (Colombia) \\
&\\
 \textsc{Julio 2012} &  {\bf Laureado del Premio Gruber para Jóvenes Investigadores}\\
 \textsc{Octubre 2011} & Departamento de Astronomía, Universidad de California, Berkeley (EEUU) \\
&\\
 \textsc{Septiembre 2011} &  {\bf Investigador postdoctoral}\\
 \textsc{Septiembre 2007} & Instituto Leibniz de Astrofísica en Potsdam (Alemania) \\ 
\end{tabular}



\color{red}
\subsection{Experiencia Organizacional}
\color{black}


\begin{tabular}{rl}	

\textsc{Actualidad} & Miembro del Centro de Investigaci\'on y Formaci\'on en   \\
\textsc{Enero 2020} & Inteligencia Artificial (CINFONIA) de la Universidad de los Andes.\\
& \\ 

\textsc{Actualidad} & Miembro del comité científico de la comunidad de Astrónomos de Colombia (AstroCO)  \\
\textsc{Enero 2019} & asociado a la Academia Colombiana de Ciencias Exactas, Físicas y Naturales.\\
& \\ 
\textsc{Actualidad} & Organizador de escuelas y congresos nacionales e internacionales\\%cycle
\textsc{Agosto 2009} & para divulgación y enseñanza de la astronomía (para aficionados y profesionales).\\
& \\ % observing program\\
  \textsc{Octubre 2020} &  Coordinador de la Oficina Andina de Astronomía para el Desarrollo\\
 \textsc{Octubre 2015} & apoyada por la Unión Astronómica Internacional.\\
& \\
\end{tabular}


\newpage

\color{red}
\subsection{Publicaciones}
\color{black}

47 publicaciones arbitradas en el \'area de astrof\'isica, cosmolog\'ia e inteligencia artificial.\\
\href{https://ui.adsabs.harvard.edu/search/filter_property_fq_property=AND&filter_property_fq_property=property%3A%22refereed%22&fq=%7B!type%3Daqp%20v%3D%24fq_property%7D&fq_property=(property%3A%22refereed%22)&q=%20author%3A%22forero-romero%22&sort=date%20desc%2C%20bibcode%20desc&p_=0} {Enlace a la lista completa de publicaciones en el SAO/NASA Astrophysics Data System}\\
\href{https://scholar.google.com/citations?user=TLTK6WgAAAAJ&hl=en}{Enlace a la lista completa de publicaciones en Google Scholar}

\color{black}



\color{red}
\subsection{Publicaciones Importantes}
\color{black}
\begin{itemize}


 \item {\it Classifying image sequences of astronomical transients
   with deep neural networks}, 
   C. G\'omez, M. Neira,  M. Hernández Hoyos, P. Arbeláez,
   {\bf J. E. Forero-Romero}, MNRAS, 499, 3, 2020.
   
\item {\it Dark matter halo shapes in the Auriga simulations}, 
  J. Prada, {\bf J. E. Forero-Romero}, R. J. J. Grand, R. Pakmor, 
  V. Springel,  MNRAS, 490, 4877, 2019.

\item{\it The Local Group in the cosmic web}, {\bf
  J.E. Forero-Romero}, R. Gonz\'alez, ApJ, 799, 1, 2015.

\item{\it Cosmic web alignments with the shape, angular momentum
  and peculiar velocities of dark matter halos}, 
  {\bf J.E. Forero-Romero}, S. Contreras, N. Padilla, MNRAS, 443,
  1090, 2014. 

\item  {\it A Dynamical Classification of the  Cosmic Web}.  {\bf
  J. E. Forero-Romero} , Y. Hoffman,  S. Gottloeber, A. Klypin, G. Yepes, 
MNRAS, 396, 1815-1824, 2009   

\end{itemize}


\color{red}
\subsection{Idiomas}
\color{black}


Español (lengua materna), Inglés (fluido), Francés (fluido), Alemán (competencia profesional básica).
 

\end{document}

