%%%%%%%%%%%%%%%%%%%%%%%%%%%%%%%%%%%%%%%%%
% Plasmati Graduate CV
% LaTeX Template
% Version 1.0 (24/3/13)
%
% This template has been downloaded from:
% http://www.LaTeXTemplates.com
%
% Original author:
% Alessandro Plasmati (alessandro.plasmati@gmail.com)
%
% License:
% CC BY-NC-SA 3.0 (http://creativecommons.org/licenses/by-nc-sa/3.0/)
%
%%%%%%%%%%%%%%%%%%%%%%%%%%%%%%%%%%%%%%%%%

%----------------------------------------------------------------------------------------
%	PACKAGES AND OTHER DOCUMENT CONFIGURATIONS
%----------------------------------------------------------------------------------------

\documentclass[a4paper,10pt]{article} % Default font size and paper size

\usepackage{fontspec} % For loading fonts
\defaultfontfeatures{Mapping=tex-text}
%\setmainfont[SmallCapsFont = Fontin SmallCaps]{Fontin} % Main document font

\usepackage{xunicode,xltxtra,url,parskip} % Formatting packages

\usepackage[usenames,dvipsnames]{xcolor} % Required for specifying custom colors

\usepackage[big]{layaureo} % Margin formatting of the A4 page, an alternative to layaureo can be \usepackage{fullpage}
% To reduce the height of the top margin uncomment: \addtolength{\voffset}{-1.3cm}

\usepackage{hyperref} % Required for adding links	and customizing them
\definecolor{linkcolour}{rgb}{0,0.2,0.6} % Link color
\hypersetup{colorlinks,breaklinks,urlcolor=linkcolour,linkcolor=linkcolour} % Set link colors throughout the document

\usepackage{titlesec} % Used to customize the \section command
\titleformat{\section}{\Large\scshape\raggedright}{}{0em}{}[\color{black} \titlerule] % Text formatting of sections

\titleformat{\subsection}{\large\scshape\raggedright}{}{0em}{}[\color{black} \titlerule] % Text formatting of sections

\titleformat{\subsubsection}{\scshape\raggedright}{}{0em}{} % Text formatting of sections


\titlespacing{\section}{0pt}{4pt}{4pt} % Spacing around sections

\begin{document}

\pagestyle{empty} % Removes page numbering



%----------------------------------------------------------------------------------------
%	PERSONAL INFORMATION
%----------------------------------------------------------------------------------------

\par{\centering{\Huge \color{black}  Jaime E. FORERO-ROMERO} \bigskip\par} % Your name

\color{red}
\section{Personal Information}
\color{black}


\begin{tabular}{rl}
\textsc{Email:} & \href{mailto:je.forero@uniandes.edu.co}{je.forero@uniandes.edu.co}\\
\textsc{Website:} & \url{http://wwwprof.uniandes.edu.co/~je.forero}\\
\textsc{GitHub:} & \url{github.com/forero}\\
\textsc{Citizenship:} & Colombian\\
\textsc{Birth Date:} & October 14, 1981\\
\end{tabular}


%----------------------------------------------------------------------------------------
%	EDUCATION
%----------------------------------------------------------------------------------------

\color{red}
\section{Research Interests}
\color{black}
\begin{itemize}
\item{Cosmology} 
\item{Galaxy Formation}
\item{The Dark Energy Spectroscopic Instrument (Collaboration Member)}
\end{itemize}


\color{red}
\section{Education}
\color{black}


\begin{tabular}{rl}	
\textsc{November 2007} & {\bf PhD. Physics}\\
\textsc{August 2005}& at the Ecole Normale Sup\'erieure de Lyon (France) \\
&\\
\textsc{July 2005} & {\bf M.Sc.Physics (Magist\`ere Interuniversitaire)}\\  
\textsc{August 2003}& at the Ecole Normale Sup\'erieure (Paris, France) \\
&\\
\textsc{July 2003} & {\bf Undergraduate Physics studies} (3$^{rd}$ \& 4$^{th}$ year)\\
\textsc{August 2001} & at the Instituto Balseiro (Argentina)\\
&\\
\textsc{July 2001}  & {\bf Undergraduate Physics studies} (1$^{st}$ \& 2$^{nd}$ year)\\
\textsc{July 1999}& at the U. Nacional de Colombia\\
\end{tabular}

\color{red}
\section{Academic Positions}
\color{black}

\begin{tabular}{rl}	
 \textsc{Present Day} &  {\bf Associate Professor}\\
 \textsc{August 2015} & Physics Department, Universidad de los Andes (Colombia) \\
& \\
 \textsc{July 2015} &  {\bf Assistant Professor}\\
 \textsc{August 2012} & Physics Department, Universidad de los Andes (Colombia) \\
&\\
 \textsc{July 2012} &  {\bf Gruber Fellow}\\
 \textsc{October 2011} & Astronomy Department UC Berkeley (USA) \\
&\\
 \textsc{September 2011} &  {\bf Postdoctoral Researcher}\\
 \textsc{September 2007} & Leibniz Insititute for Astrophysics at Potsdam(Germany) \\ 
&\\
 \textsc{August 2007} &  {\bf Graduate Student}\\
 \textsc{August 2006} & Ecole Normale Sup\'erieure de Lyon (France)\\
&\\
 \textsc{July 2006} &  {\bf \'El\`eve}\\
 \textsc{September 2003} & Ecole Normale Sup\'erieure (Paris, France)\\
&\\
 \textsc{August 2003} &  {\bf Undergraduate Fellow}\\
 \textsc{August 2001} & Instituto Balseiro (Bariloche, Argentina)\\
\end{tabular}

\color{red}
\section{Research}
\color{black}

\subsection{Bibliometrics}
\begin{itemize}
\item Refereed journal articles:  26 published + 1 accepeted + 1 submitted (since
  first paper in 2007).
\item Google Scholar statistics. Total Number of Citations: 728. H-index: 14.
\item SAO/NASA Astrophysics Data System statistics. Total Number of Citations: 604. H-index: 13.
\end{itemize}
\subsection{Refereed Publications}
\begin{enumerate}

\subsubsection{2017}

\item[28]{\it Modelling the gas kinematics of an atypical Lyman alpha
emitting compact dwarf galaxy}, {\bf J. E. Forero-Romero}, M. Gronke, M. C. Remolina-Guti\'errez,
Nicol\'as Garavito-Camargo, Mark Dijkstra, MNRAS submitted. 

\item[27]{\it Tracing the cosmic web}, N. I. Libeskind, R. van de Weygaert, M. Cautun, B. Falck, E.
Tempel, T. Abel, M. Alpaslan, M. A. Aragón-Calvo, {\bf
  J. E. Forero-Romero},  R. Gonzalez, S. Gottl\"ober, O. Hahn 13 ,
W. A. Hellwing, Y. Hoffman, B. J. T. Jones, F. Kitaura, A. Knebe,
S. Manti, M. Neyrinck, S. E. Nuza, N. Padilla, E. Platen,
N. Ramachandra, A. Robotham, E. Saar, S. Shandarin, M. Steinmetz,
R. S. Stoica, Th. Sousbie, G. Yepes, MNRAS accepted.  


\subsubsection{2016}

\item[26]{\it Boosting Lya and HeII 1640A Line Fluxes from Pop III
  Galaxies: Stochastic IMF Sampling and Departures from
  Case-B}. L. Mas-Ribis, M. Dijkstra, {\bf J.E. Forero-Romero},
  ApJ, 833, 1, 2016.

\item[25]{\it Quantifying and controlling biases in dark matter halo
  concentration estimates}, C.N. Poveda-Ruiz, {\bf
  J.E. Forero-Romero}, J.C. Mu\~noz-Cuartas, ApJ, 832, 2, 2016. 

\item[24]{\it Impact of Cosmic Variance on the Galaxy-Halo Connection
  for Lyman-$\alpha$ emitters}.  J.E. Mej\'ia-Restrepo, {\bf
  J.E. Forero-Romero}, ApJ, 821, 1, 2016

\item[23]{\it SPOKES: An end-to-end simulation facility for
  spectroscopic cosmological surveys}, 
	Nord, B.; Amara, A.; R\'efr\'egier, A.; Gamper, La.; Gamper, Lu.;
        Hambrecht, B.; Chang, C.; {\bf Forero-Romero, J. E.}; Serrano, S.;
        Cunha, C.; Coles, O.; Nicola, A.; Busha, M.; Bauer, A.;
        Saunders, W.; Jouvel, S.; Kirk, D.; Wechsler, R., Astronomy
        and Computing, 15, 1, 2016


\subsubsection{2015}
\item[22]{\it Tensor anisotropy as a tracer of cosmic voids},
  S. Bustamante, {\bf   J.E. Forero-Romero}, MNRAS, 453,
  497, 2015

\item[21]{\it The Local Group in the cosmic web}, {\bf
  J.E. Forero-Romero}, R. Gonz\'alez, ApJ, 799, 1, 2015.

\subsubsection{2014}
\item[20]{\it Cosmological constraints from the redshift dependence of
  the Alcock-Paczynski effect in the galaxy density gradient field},
  X-D. Li, C. Park, {\bf J.E. Forero-Romero}, J. Kim, ApJ, 796, 2,
  2014.


\item[19]{\it The impact of gas bulk rotation on the lyman-alpha line}
  J.N. Garavito-Camargo,  {\bf J.E. Forero-Romero}, M. Dijkstra, ApJ,
  795, 2, 2014. 
 

\item[18]{\it  Characterizing SL2S galaxy groups using the Einstein
  radius}, T. Verdugo, V. Motta, G. Fo\"ex, {\bf J. E. Forero-Romero},
  R. P. Mu\~noz, R. Pello, M. Limousin, A. More, R. Cabanac, G. Soucail,
  J. P. Blakeslee, A. J. Mej\'ia-Narv\'aez, G. Magris,
  J. G. Fern\'andez-Trincado, Astronomy \& Astrophysics, 571, id.A65, 2014.


\item[17]{\it Systematic uncertainties from halo asphericity in dark
  matter searches}  N. Bernal, {\bf J.E. Forero-Romero}, R. Garani,
  S. Palomares-Ruiz, JCAP, 09, 004, 2014.

\item[16]{\it Cosmic web alignments with the shape, angular momentum
  and peculiar velocities of dark matter halos}, 
  {\bf J.E. Forero-Romero}, S. Contreras, N. Padilla, MNRAS, 443,
  1090, 2014. 

\item[15]{\it The abundance of Bullet Groups in $\Lambda$CDM},
  J. G. Fern\'andez-Trincado, {\bf J. E. Forero-Romero}, G. Foex,
  V. Motta, T. Verdugo, V. Motta, ApJ Letter, 787, L32, 2014.

\subsubsection{2013}
\item[14]{\it The MultiDark Database: Release of the Bolshoi and
  MultiDark Cosmological Simulations} , K. Riebe , A. M. Partl,
  H. Enke, {\bf J.E. Forero-Romero}, S. Gottloeber, A. Klypin,
  G. Lemson, F. Prada, J. R. Primack, M. Steinmetz, V. Turchaninov,
  Astronomische Nachrichten, 334, 691, 2013. 

\item[13] {\it The kinematics of the Local Group in a cosmological context}, 
{\bf J.E. Forero-Romero}, Y. Hoffman, S. Bustamante, S. Gottloeber,
G. Yepes, ApJ Letters, 767, 1, 2013 

\item[12] {\it The velocity shear tensor: tracer of halo alignment},
  Libeskind N., Hoffman Y., {\bf J.E. Forero-Romero} , S. Gottloeber,
  A. Knebe, M. Steinmentz, A. Klypin, MNRAS 428, 2489, 2013 

\item[11] {\it Effects of Star Formation Stochasticity on the Ly
  $alpha$ \& Lyman Continuum Emission from Dwarf Galaxies}, {\bf
  J. E. Forero-Romero} \& M. Dijkstra, MNRAS 428, 2163, 2013 

\subsubsection{2012}
\item[10] {\it A kinematic classification of the cosmic web},
  Y. Hoffman, O. Metuki , G. Yepes, S. Gottloeber, {\bf
    J. E. Forero-Romero}, N. I. Libeskind, A. Knebe, MNRAS, 425, 2049,
  2012 

\item[9] {\it Modelling the fraction of Lyman Break Galaxies with
  strong Lyman-alpha emission at $5 < z < 7$} {\bf Forero-Romero
  J.E.}, Yepes G., Gottloeber S., Prada F., MNRAS, 419, 952, 2012 

\subsubsection{2011}
\item [8]
{\it The dark matter assembly of the Local Group in constrained cosmological
  simulations of a $\Lambda$CDM universe} {\bf Forero-Romero} J.E.,
Hoffman Y., Yepes G., Gottl\"ober S., Piontek R., Klypin A., Steinmetz
M.,  MNRAS, 417, 1434, 2011 

\item[7] 
{\it Halo based reconstruction of the cosmic mass density field}
Mun\~oz-Cuartas J. C., M\"uller V, {\bf Forero-Romero} J. E.,
MNRAS, 417, 1303, 2011 
 
\item [6]
{\it {\tt CLARA}'s view on the escape fraction of Lyman-$\alpha$ photons in
  high redshift galaxies}
{\bf Forero-Romero} J.E., Yepes G., Gottl\"ober S., Knollmann S., Cuesta A., Prada F.,  
MNRAS, 415, 3666, 2011

\subsubsection{2010}
\item [5]
{\it Bullet Clusters in the MareNostrum Universe}. 
{\bf Forero-Romero} J.E., Yepes G., Gottl\"ober S., 
ApJ, 725, 1, 2010.

\item [4]
{\it Simulated vs. observed UV emission at high redshift: a hint for a clumpy
ISM? }. 
{\bf Forero-Romero} J.E., Yepes G., Gottl\"ober S., Knollmann S., Khalatyan A., Cuesta A., Prada F.,   MNRAS Letters, 403,  L31-L35, 2010

\subsubsection{2009}
\item [3]
{\it The coarse geometry of merger trees in
  $\Lambda$CDM}.  {\bf Forero-Romero} J.E., 
MNRAS, 399, 762-768, 2009

\item [2]
{\it A Dynamical Classification of the  Cosmic Web}.  {\bf Forero-Romero} J.E., Hoffman Y.,  Gottloeber S., Klypin A., Yepes G.,
MNRAS, 396, 1815-1824, 2009

\subsubsection{2007}
\item [1] 
{\it {\sc LeMoMaF}:  Lensed Mock Map Facility}. 
{\bf Forero-Romero} J.E., Blaizot J., Devriendt J., Van Waerbeke L., Guiderdoni B., 
MNRAS, 379. 1507-1518, 2007
\end{enumerate}

%\subsection{ArXiv-Only Publications}
\subsection{Supervised Postdocs}

\begin{itemize}
\item 2015-2016. Ver\'onica Arias. Dynamics of satellites in the
  Local Group.  
\end{itemize}

\subsection{Supervised Students}

\begin{itemize}
\subsubsection{2017}
\item[8] Sebasti\'an Franco Ulloa. 
  Undegraduate thesis in Physics at Universidad de los Andes. 
\emph{Lattice Boltzmann methods to simulate dark matter.}

\subsubsection{2016}
\item[7]
  Nicol\'as Romero D\'iaz.
  Undegraduate thesis in Physics at Universidad de los Andes. 
  \emph{Observational evidence of star formation stochasticity in the
    CALIFA dataset.}
\item[6] 
  David Esteban  Bernal Neira.
  Undegraduate thesis in Physics at Universidad de los Andes. 
  \emph{Acotando las velocidades tangenciales de las galaxias
    sat\'elite de Andr\'omeda utilizando optimizaci\'on no lineal.} 
\item[5] Sergio Hern\'andez Charpak. Undegraduate
  thesis in Physics at Universidad de los Andes. \emph{Laniakea in a cosmological context}. 
\subsubsection{2015}
\item[4] Mar\'ia Camila Remolina Guti\'errez. Undegraduate
  thesis in Physics at Universidad de los Andes. \emph{The joint
    effect of outflows and rotation on the Lyman-alpha line}. 
\item[3] Juan Nicol\'as Garavito Camargo. Master
  thesis in Physics at Universidad de los Andes. \emph{The effect of
    gas bulk rotation on the Lyman-alpha line}.
\item[2] Christian Nicanor Poveda Ruiz. Undegraduate thesis
  in Physics at Universidad de los Andes. \emph{A semi-analytic
    approach to formation processes in galaxies}. 
\subsubsection{2014}
\item[1] Camilo Andr\'es Rivera Lozano. Undergraduate thesis
  in Physics at Universidad de los Andes. \emph{Impacto de los
    par\'ametros cosmol\'ogicos en la 
    estructura a gran escala del Universo}.
\end{itemize}

\subsection{Grants}
\begin{itemize}
\item 2017-2021 (coPI) LACEGAL Latin American Chinese European GALaxy Formation network.
Funded by MSCA-RISE - Marie Sklodowska-Curie Research and Innovation Staff Exchange (RISE). (1.5 MEuro)
\item 2016-2020 (PI) COLCIENCIAS grant to work on the Dark
  Energy Spectroscopic Instrument. (60 KUSD)
\item 2012-2016 (PI) UNIANDES grant given as a startup package to
assistant professors (25 KUSD).
\item 2011-2012 (PI) IAU Peter and Patricia Gruber Foundation
  Fellowship. (50K USD).
\end{itemize}

%\subsection{Talks}

%\color{red}
%\section{Teaching}
%\color{black}
%\subsection{Lectures at Los Andes University}
%\subsection{Summer/Winter Schools}



\color{red}
\section{Service}
\color{black}

\subsection{International Organizing Committees}
\begin{tabular}{rl}	
 \textsc{Present Day} &  Coordinator for the Andean Regional Office of  Astronomy for Development.\\
 \textsc{October 2015} & \url{http://andean.astro4dev.org/}\\
& \\
 \textsc{October 2015} &  Main organizer of the Second
 \textit{Workshop Astronom\'ia en los Andes} \\
   & to convene astronomers in the Andean region.  Held in Bogot\'a, Colombia.\\
& \\
 \textsc{July 2013} &  Main organizer of the First
 \textit{Workshop Astronom\'ia en los Andes} \\
   & to convene astronomers in the Andean region.  Held in Bogot\'a, Colombia.\\
& \\
 \textsc{September 2011} &  Co-organizer of the international
 conference \textit{Young and Bright:}\\
 & \textit{understanding high redshigt structures}. Held in
 Potsdam, Germany.\\ 
\end{tabular}


%\subsection{National Organizing Committees}
%\subsection{International Refereeing}
%\subsection{National Refereeing}
%\subsection{Jury of Master/PhD Thesis}
%\subsection{Public Talks}



\color{red}
\section{Miscellanea}
\color{black}
%\subsection{Art \& Science}


%\begin{itemize}
%\subsubsection{2015}
%\item Collaboration with visual artist M\'onica Naranjo to
%  produce a \emph{Cosmological Calendar}. 
%\subsubsection{2014}
%\item Collaboration with architect and plastic artist Carlos
%  Moncada on the project \emph{Paisajes Estereoplanetarios} as part of
%  the \emph{Astronom\'ia Perif\'erica}. 
%\subsubsection{2013}
%\item Founding member of the project \emph{Astronom\'ia
%  Perif\'erica} that aims at taking astronomy to Bogota's periphery
%  through artistic interventions.  Secured funding for $\sim 5$KEuro
%  from the Office of Astronomy for Development to kickstart the
%  project.  
%\subsubsection{2012}
%\item Founding member and organizer of the Art and Science
%  Meetings (Encuentros de Arte y Ciencia) in Bogot\'a.  
%\subsubsection{2011}
%\item {\it Fluid Skies} secured funding for $50$ KEuro
%  to produce the Carved Air exhibition (Berlin, 2012), print the
%  exhibition catalog and prepare academic meetings on the scientific
%  and historical concepts of the project. The sponsors were: Ernst
%  Schering Foundation (Germany), University of California Institute
%  for Research in the Arts (USA) and the Arts Council Korea (ARKO).  
%\item Founding member of the {\it Fluid Skies} collaboration
%  together with Yunchul Kim (artist) and Lucia Ayala (Art
%  Historian). {\it Fluid Skies} is a creative and research platform to
%  explore the fluid materiality of the cosmos from the perspectives of
%  astrophysics, art, history and philosophy. 
%\subsubsection{2010}
%\item Talk on {\it Simulations of Large Scale Structure
%  Evolution}, Institut f\"ur Raumexperimente - Olafur Eliasson Studio,
%  Berlin.  
%\end{itemize}


\subsection{Languages}

\begin{tabular}{rl}
\textsc{Spanish:} & Native\\
\textsc{English:} & Fluent\\
\textsc{French:} & Fluent \\
\textsc{German:} & Proficient \\
\textsc{Russian:} & Beginner \\
\end{tabular}

\end{document}



\end{document}

\begin{tabular}{rl}	
\textsc{December 2016} & \textbf{Physics \& Systems and Computing Engineering Double Major} \\ 
\textsc{August 2012} & at Universidad de los Andes in Bogot\'a, Colombia. GPA: 4.45/5.0.  \\ 
 & \url{fisica.uniandes.edu.co/} \& \url{sistemas.uniandes.edu.co/} \\
&\\

\textsc{December} 2011 & \textbf{Bachelor} from Fundaci\'on Colegio UIS in Bucaramanga, Santander, \\
\textsc{January} 2001 & Colombia. \url{fcuis.edu.co/}\\

\end{tabular}


%----------------------------------------------------------------------------------------
%	RESEARCH
%----------------------------------------------------------------------------------------

\color{OrangeRed}
\section{Research Experience}
\color{black}

All the projects listed below have a wider explanation at my website under the following url: \url{mariacamilaremolinagutierrez.github.io/research}\\

\begin{tabular}{rl}
\textsc{Present Day} & \textbf{Summer Undergraduate Research Fellowship at NASA's Jet} \\
\textsc{June 2016} & \textbf{Propulsion Laboratory}. Title: Analysis of the size-flux relationship \\
& of the Point-Spread Function of near infrared detectors for gravitational \\ 
& weak lensing measurements. Advisors: Dr. Jason Rhodes, Dr. Andr\'es \\
& Plazas and Dr. Charles Shapiro. \\
&\\

\textsc{May 2016} & \textbf{Undergraduate physics thesis}. Title: Study of the influence of \\
\textsc{February 2014} & galaxy rotation and outflows in the Lyman Alpha spectral line. Advisor: \\
& Dr. Jaime Forero-Romero.\\
&\\

\textsc{December 2015} & \textbf{Research Assistant for a Ph.D. thesis}. Title: Astronomical image \\ 
\textsc{August 2015} & processing from large all-sky photometric surveys for the detection and \\
& the measurement of transients. Ph.D. Candidate: Juan Pablo Reyes. \\
& Advisor: Dr. Marcela Hern\'andez Hoyos. \\
&\\

\textsc{August} 2015 & \textbf{Summer Astrophysics ESA Program at Leiden Observatory} \\ 
\textsc{June} 2015 & Title: Ballet of the Free-Floating Planets. Advisor: Dr. Lucie J\'ilkov\'a. \\
\end{tabular}


%----------------------------------------------------------------------------------------
%	COMPUTER SKILLS 
%----------------------------------------------------------------------------------------

\color{OrangeRed}
\section{Computer Skills}
\color{black}
 
\begin{tabular}{rl}
Programming Languages: & Python (P), C (P), Java (P), Javascript (I), HTML (B).\\
Software: & GitHub, {\LaTeX}, \textsc{UNIX}.\\
Operating Systems: & Ubuntu, Windows, Mac.\\
&\\
Proficiency Levels: & P = Proficient, I = Intermediate, B = Basic.
\end{tabular}

% NEW PAGE
\newpage
% NEW PAGE

%----------------------------------------------------------------------------------------
%	SCHOOLS AND EVENTS
%----------------------------------------------------------------------------------------

\color{OrangeRed}
\section{Talks, Posters and Schools}
\color{black}

\begin{tabular}{rl}
\textsc{April} 2016 & Congress: Escape of Lyman radiation from galactic labyrinths  in Crete,\\ & Greece. Short talk.\\
&\\
\textsc{December} 2014 & Congress: IV Colombian Congress of Astronomy (COCOA) in Pasto,\\ & Colombia. Poster presentation.\\
&\\
\textsc{November} 2013 & Meeting: XIV Latin American Regional IAU Meeting (LARIM) in \\ & Florianopolis, Brazil. Poster presentation.\\
&\\
\textsc{July} 2013 & Workshop: Astronomía en los Andes (Astronomy in the Andes) in \\
& Bogota, Colombia. Short talk.\\
&\\
\textsc{June} 2011 & Summer Camp: Summer Science Program in Socorro, New Mexico,\\
& United States of America. \url{http://www.summerscience.org}\\
\end{tabular}


%----------------------------------------------------------------------------------------
%	SCHOLARSHIPS AND DISTINCTIONS
%----------------------------------------------------------------------------------------

\color{OrangeRed}
\section{Scholarships and Distinctions}
\color{black}

\begin{tabular}{rl}
\textsc{December} 2017 & \textbf{Quiero Estudiar Scholarship} for academic excellence that covers\\
\textsc{July} 2012&  tuition for undergraduate studies at Universidad de los Andes. \\ 
&\\
\textsc{November} 2011 & \textbf{Position 62} nationally in the colombian standarized tests (ICFES).\\ 
&\\
\textsc{November} 2011 & \textbf{Regional Andres Bello} distinction for the best mathematics score\\
& in Santander at the colombian standarized tests (ICFES).\\
&\\
\textsc{2007-2011} & Individual and regional representant at the \textbf{Colombian Mathematics}\\
& \textbf{Olympiads}. \url{http://oc.uan.edu.co/ocm/om98a.aspx} \\ 
&\\
\textsc{2006-2011} & \textbf{Honour Tution} for academic excellence in high school.\\
\end{tabular}


%----------------------------------------------------------------------------------------
%	SERVICE
%----------------------------------------------------------------------------------------

\color{OrangeRed}
\section{Science Volunteering}
\color{black}

\begin{tabular}{rl}

\textsc{July 2013} & Assistant in the creation of the \textbf{Astronomy Office for} \\
\textsc{January 2013} & \textbf{Development} for the Andean region.\\
&\\
\textsc{November} 2011 & Creation of RECA (Network of Colombian Astronomy Students).\\ 
& \url{https://sites.google.com/site/recastronomia/}\\
&\\
\textsc{2010-2011} & Director and editor of the scientific magazine \textbf{QUANTOS}.
\end{tabular}


%----------------------------------------------------------------------------------------
%	LANGUAGES
%----------------------------------------------------------------------------------------

\color{OrangeRed}
\section{Languages}
\color{black}

\begin{tabular}{rl}
\textsc{Spanish:} & Mother tongue.\\

\textsc{English:} & Fluent. (TOEFL iBT Score: 101/120)\\

\textsc{German:} & Learning. \\
\end{tabular}


%----------------------------------------------------------------------------------------
%	EXTRACURRICULAR ACTIVITIES
%----------------------------------------------------------------------------------------

\color{OrangeRed} 
\section{Extracurricular Activities}
\color{black}

\begin{itemize}
    \item Environmental Radio Broadcasting.
    \item Choir singing. Bongo drum.
    \item Squash, Volleyball, Swimming.
\end{itemize}

%----------------------------------------------------------------------------------------

\end{document}

%Lastly modified: July the 27th, 2016
