\documentclass[9pt]{article}
\newcommand{\galics}{{\sc GalICS}}
\newcommand{\momaf}{{\sc MoMaF}}
\newcommand{\lemomaf}{{\sc LeMoMaF}}
%\textheight=25.5cm
%\textwidth=16.0cm
%\oddsidemargin=-0.5cm
%\topmargin=-2.0cm
\pagenumbering{arabic}
\bibliographystyle{unsrt}

\begin{document}

\noindent

%==================================WHOAMI===============================================
%\colorbox{gris-claro}{{\Large Jaime Ernesto FORERO ROMERO}}\\
\noindent
{{\Large Jaime E. FORERO-ROMERO}}\\
%=================================NACIMIENTO============================================
\begin{flushleft}
Born October 14, 1981. Bogot\'a, Colombia\\
\end{flushleft}
%==================================DIRECCION============================================
\begin{flushright}
\verb"http://wwwprof.uniandes.edu.co/~je.forero/" \\{\tt{je.forero at uniandes dot edu dot co}}\\
\verb"https://github.com/forero"\\
Calle 18A\# 1 - 10\\
Bloque Ip, Of. 208\\
Universidad de los Andes\\
AA 4976, Bogot\'a, Colombia\\
\end{flushright}


{\bf Main Research Interests}
\begin{itemize}
\item[-]{High Redshift Galaxy Evolution}
\item[-]{Milky Way Assembly}
\item[-]{Cosmic Web Characterization}
\end{itemize}

{\bf Education}
\indent
\begin{itemize}
\item[-] PhD. Physics, Ecole Normale Sup\'erieure de Lyon (France) 11/2007
\item[-] M.Sc.Physics (Magistère Interuniversitaire)  Ecole Normale
  Sup\'erieure (Paris, France), 08/2005
\item[-] Undergraduate Physics studies (3$^{rd}$-4$^{th}$ years),
  Instituto Balseiro (Argentina), 08/2001 - 08/2003
\item[-] Undergraduate Physics studies (1$^{st}$-2$^{nd}$ years),
  U. Nacional de Colombia, 07/1999 -  07/2001
\end{itemize}

{\bf Academic Positions}
\begin{itemize}
\item[-] Assistant Professor, Physics Department, Universidad de los Andes (Colombia), 8/2012-
\item[-] Gruber Fellow at the Astronomy Department UC Berkeley (USA), 10/2011-7/2012
\item[-] Postdoctoral Researcher at the Leibniz Institute for
  Astrophysics-Potsdam (AIP) (Germany), 10/2007-9/2011  
\item[-] Graduate student at the Ecole Normale Sup\'erieure de Lyon
  (France), 09/2006-09/2007  
\item[-] \'El\`eve at the Ecole Normale Sup\'erieure (Paris, France),
  09/2003-08/2006 
\item[-] Undergraduate fellow at the Instituto Balseiro (Bariloche,
  Argentina), 08/2001-08/2003 
\end{itemize}

\newpage
{\bf Honors and Awards}
\begin{itemize}
\item[-] 2011. IAU Peter and Patricia Gruber Foundation Fellowship
\item[-] 2006-2007. Graduate Research Fellowship (Allocation Coupl\'ee) of the French
  Ministry for Research and Education.
\item[-] 2003-2006. Scholarship at the Ecole Normale Sup\'erieure (Paris), covers tuition fees plus room and
  board to pursue graduate studies and PhD research.
\item[-] 2001-2003. Undergraduate scholarship (covers tuition fees at the
  Instituto Balseiro plus room and board) from the Argentinian Atomic Energy
  Comission.
\item[-] 1999. Gold Medal in the National Chemistry Olympiad, Bogot\'a,
  Colombia (First Place, National Competition)
\end{itemize}

{\bf Grants}
\begin{itemize}
\item[-] 2013. (PI) Universidad de los Andes grant to organize the
  international Workshop \emph{Astronom\'ia en los Andes} to set up a
  collaboration network in the Andean region (5K USD). 
\item[-] 2013. (Co-PI) ICTP grant to organize the international
  Workshop \emph{Astronom\'ia en los Andes} to set up a collaboration
  network in the Andean region (6K USD). 
\item[-] 2013. (PI) Astronomy for Development Grant for the outreach
  project \emph{Astronom\'ia Perif\'erica} (5K USD). 
\item[-] 2012. (PI) NVIDIA Academic Partnership, Inkind donation of 1
  TESLA GPU (worth $\sim$2K USD). 
\item[-] 2011. (PI) IAU Peter and Patricia Gruber Foundation
  Fellowship. (50K USD).
\end{itemize}

{\bf Service}
\begin{itemize}
\item[-] 2013. Main organizer of the Worskhop {\textit Astronom\'ia en
  los Andes} to convene astronomers in the Andean region.  
\item[-] 2013-present. In charge of the Astronomy Undergraduante and Graduate Seminar at Uniandes. 
\item[-] 2012. Co-organizer and SOC member, 3rd Colombian Congress of
  Astronomy (Bucaramanga) 
\item[-] 2011. Co-organizer of the international conference {\it Young
  and Bright: understanding high redshigt structures} ($\sim 60$
  attendees), Potsdam. 
\item[-] 2014-present. Referee for JCAP  %(1).  
\item[-] 2013-present. Referee for the Astrophysical Journal %(1).  
\item[-] 2011-present. Referee for Monthly Notices of the Royal
  Astronomical Society %(3).  
\item[-] 2010-present. Referee for the Colombian funding agency {\it
  COLCIENCIAS} %(1). 
\item[-] 2010-present. Referee for {\it Revista Colombiana de
  F\'{\i}sica} %(2).  
\item[-] 2010/2011. Co-organizer of internal workshops to give an
  institute wide research update in the extragalactic astronomy branch
  ($\sim 30$ attendees), Potsdam. 
\item[-] 2010. SOC member, 2nd Colombian Congress of Astronomy (Bogot\'a).
\item[-] 2008. Co-organizer and SOC member, 1st Colombian Congress of
  Astronomy (Medell\'{\i}n). 
\end{itemize}

{\bf Teaching}
\begin{itemize}

\item[-] {\it Physics II (Electromagnetism)}. Universidad de
  los Andes, 2014-I, Student's rating: 3.55/4.00 (60 students).  
\item[-] {\it Computational Methods (Basic)}. Universidad de los
  Andes., 2014-I, Student's rating: 3.76/4.00 (12 students).
\item[-] {\it Physics II (Electromagnetism)}. Universidad de
  los Andes, 2013-II: 3.47/5.00 (61 students).
\item[-] {\it Computational Physics (Basic)}. Universidad de los
  Andes., 2013-II, Student's rating: 3.36/4.00 (28 students).
\item[-] {\it Physics I (Newtonian Mechanics)}. Universidad
  de los Andes., 2013-I, Student's rating: 3.52/4.00. (68 students).
\item[-] {\it Computational Physics (Basic)}. Universidad de los
  Andes., 2013-I, Student's rating: 3.90/4.00 (10 students).
\item[-] {\it Physics I (Newtonian Mechanics)}. Universidad
  de los Andes., 2012-II, Student's rating: 3.41/4.00 (65 students).
\item[-] {\it Introduction to Physics (module on
  Astrophysics)}. Universidad de los Andes.  2012-II
\item[-]{\it Lectures on Galaxy Evolution}, V Colombian
  Astrophysics Summer School, Observatorio Astronomico Nacional, Summer 2009
\item[-]{\it{Introduction to Astrophysics (Teaching
      Assistant)}}, Undergraduate Level Course at the University of Lyon I,
      Spring 2007.
\end{itemize}


{\bf Students Supervised}
\begin{itemize}

\item[-] 2014-20: Camilo Andr\'es Rivera Lozano. Undergraduate thesis
  in Physics at Universidad de los Andes. \emph{Impaccto de los
    par\'ametros cosmol\'ogicos en la 
    estructura a gran escala del Universo}.
\item[-] Since August 2013: Felipe G\'omez. PhD Student at
  Universidad de los Andes. Research project on the
  influence of the cosmic web on dark matter halo formation. 
\item[-] Since October 2012: Nicol\'as Garavito. Master Student
  at Universidad de los Andes. Research project on
  radiative transfer of the  Lyman-$\alpha$ line.  
\item[-] 2013-10: Sebastian Bustamante. Undergraduate thesis in
  Physcis at Universidad de Antioquia on the large scale environment
  of the Local Group.  
\item[-] 2012-10: David Nore\~na. Undergraduate thesis at Universidad
  de Antioquia (Colombia) on tidal streams in the Milky Way. 
\item[-] 2010-10. Juli\'an Mej\'{\i}a. Undergraduate thesis at the
  Universidad de Antioquia on a halo model of
  Lyman-$\alpha$ emitters at high redshift. 
\end{itemize}


{\bf Jury of Master Thesis}
\begin{itemize}
\item[-] 2013. Roger Hurtado. Master student at the Observatorio Astron\'omico Nacional de Colombia. Thesis on gravitational lensing.
\end{itemize}

{\bf Publications}\\

\indent
Refereed Journal Articles\\

ADS statistics. Total Number of Citations: 275. H-index: 9.
\begin{enumerate}

\item[21]{\it Cosmological constraints from the redshift dependence of
  the Alcock-Paczynski effect in the galaxy density gradient field},
  X-D. Li, C. Park, {\bf J.E. Forero-Romero}, J. Kim, ApJ submitted,
  2014. 

\item[20]{\it The Local Group in the cosmic web}, {\bf
  J.E. Forero-Romero}, R. Gonz\'alez, ApJ submitted, 2014.

\item[19]{\it  Characterizing SL2S galaxy groups using the Einstein
  radius}, T. Verdugo, V. Motta, G. Foëx, {\bf J. E. Forero-Romero},
  R. P. Muñoz, R. Pello, M. Limousin, A. More, R. Cabanac, G. Soucail,
  J. P. Blakeslee, A. J. Mej\'ia-Narváez, G. Magris,
  J. G. Fernández-Trincado, A\&A accepted, 2014.

\item[18]{\it The impact of gas bulk rotation on the lyman-alpha line}
  J.N. Garavito-Camargo,  {\bf J.E. Forero-Romero}, M. Dijkstra, ApJ
  accepted, 2014. 

\item[17]{\it Systematic uncertainties from halo asphericity in dark
  matter searches}  N. Bernal, {\bf J.E. Forero-Romero}, R. Garani,
  S. Palomares-Ruiz, JCAP, 09, 004, 2014.

\item[16]{\it Cosmic web alignments with the shape, angular momentum
  and peculiar velocities of dark matter halos}, 
  {\bf J.E. Forero-Romero}, S. Contreras, N. Padilla, MNRAS, 443,
  1090, 2014. 

\item[15]{\it The abundance of Bullet Groups in $\Lambda$CDM},
  J. G. Fern\'andez-Trincado, {\bf J. E. Forero-Romero}, G. Foex,
  V. Motta, T. Verdugo, V. Motta, ApJ Letter, 787, L32, 2014.

\item[14]{\it The MultiDark Database: Release of the Bolshoi and
  MultiDark Cosmological Simulations} , K. Riebe , A. M. Partl,
  H. Enke, {\bf J.E. Forero-Romero}, S. Gottloeber, A. Klypin,
  G. Lemson, F. Prada, J. R. Primack, M. Steinmetz, V. Turchaninov,
  Astronomische Nachrichten, 334, 691, 2013. 

\item[13] {\it The kinematics of the Local Group in a cosmological context}, 
{\bf J.E. Forero-Romero}, Y. Hoffman, S. Bustamante, S. Gottloeber,
G. Yepes, ApJ Letters, 767, 1, 2013 


\item[12] {\it The velocity shear tensor: tracer of halo alignment},
  Libeskind N., Hoffman Y., {\bf Forero-Romero} J.E., Gottloeber S.,
  Knebe A., Steinmentz M., Klypin A., MNRAS 428, 2489, 2013 

\item[11] {\it Effects of Star Formation Stochasticity on the Ly
  $alpha$ \& Lyman Continuum Emission from Dwarf Galaxies}, {\bf
  J. E. Forero-Romero} \& M. Dijkstra, MNRAS 428, 2163, 2013 

\item[10] {\it A kinematic classification of the cosmic web},
  Y. Hoffman, O. Metuki , G. Yepes, S. Gottloeber, {\bf
    J. E. Forero-Romero}, N. I. Libeskind, A. Knebe, MNRAS, 425, 2049,
  2012 

\item[9] {\it Modelling the fraction of Lyman Break Galaxies with
  strong Lyman-alpha emission at $5 < z < 7$} {\bf Forero-Romero
  J.E.}, Yepes G., Gottloeber S., Prada F., MNRAS, 419, 952, 2012 

\item [8]
{\it The dark matter assembly of the Local Group in constrained cosmological
  simulations of a $\Lambda$CDM universe} {\bf Forero-Romero} J.E.,
Hoffman Y., Yepes G., Gottl\"ober S., Piontek R., Klypin A., Steinmetz
M.,  MNRAS, 417, 1434, 2011 

\item[7] 
{\it Halo based reconstruction of the cosmic mass density field}
Mun\~oz-Cuartas J. C., M\"uller V, {\bf Forero-Romero} J. E.,
MNRAS, 417, 1303, 2011 
 
\item [6]
{\it {\tt CLARA}'s view on the escape fraction of Lyman-$\alpha$ photons in
  high redshift galaxies}
{\bf Forero-Romero} J.E., Yepes G., Gottl\"ober S., Knollmann S., Cuesta A., Prada F.,  
MNRAS, 415, 3666, 2011

\item [5]
{\it Bullet Clusters in the MareNostrum Universe}. 
{\bf Forero-Romero} J.E., Yepes G., Gottl\"ober S., 
ApJ, 725, 1, 2010.

\item [4]
{\it Simulated vs. observed UV emission at high redshift: a hint for a clumpy
ISM? }. 
{\bf Forero-Romero} J.E., Yepes G., Gottl\"ober S., Knollmann S., Khalatyan A., Cuesta A., Prada F.,   MNRAS Letters, 403,  L31-L35, 2010

\item [3]
{\it The coarse geometry of merger trees in
  $\Lambda$CDM}.  {\bf Forero-Romero} J.E., 
MNRAS, 399, 762-768, 2009

\item [2]
{\it A Dynamical Classification of the  Cosmic Web}.  {\bf Forero-Romero} J.E., Hoffman Y.,  Gottloeber S., Klypin A., Yepes G.,
MNRAS, 396, 1815-1824, 2009

\item [1] 
{\it {\sc LeMoMaF}:  Lensed Mock Map Facility}. 
{\bf Forero-Romero} J.E., Blaizot J., Devriendt J., Van Waerbeke L., Guiderdoni B., 
MNRAS, 379. 1507-1518, 2007
\end{enumerate}




Non refereed 
\begin{enumerate}
\item[1] {\it Visualising Matter and Cosmologies: an Example Based on
  a Transhistorical Approach}, Lucia Ayala, Jaime E. Forero-Romero,
  Column 7, pp. 76-82 (2011) 
\end{enumerate}


{\bf Selected Talks}
\begin{itemize} 
\item [-] 2014, {\it The Local Group in the Cosmic Web}, 11th Potsdam
  Thinkshop on Satellite Galaxies and Dwarfs in the Local Group,
  Potsdam.
\item [-] 2013, {\it F\'isica fundamental y astronom\'ia: conexi\'on a trav\'es de la formaci\'on de galaxias}, Colombian Physics Conference, Colombia
\item [-] 2013, {\it Stochasticity in high-z dwarf galaxies},
  Extragalactic Group Seminar, PUC, Santiago
\item [-] 2012, {\it New Topics on High-z Galaxy formation}, Cosmology
  Group Seminar, MPA, Garching 
\item [-] 2012, {\it Towards a panchromatic picture of high-z
  galaxies}, Cosmology Group Seminar, Stanford 
\item [-] 2011, {\it Building a panchromatic model of high-z
  galaxies}, Second Workshop on Numerical and Observational
  Astrophysics: From the First Structures to the Universe Today,
  Institute for Astronomy and Space Physics (IAFE), Buenos Aires. 
\item [-] 2011, {\it Expanding universes: the evolution of numerical
  simulations in physical cosmology}, Computer Simulations and the
  Changing Face of Scientific Experimentation, University of
  Stuttgart, Stuttgart. 
\item [-] 2011, {\it Towards a panchromatic model of high-z galaxies}, First
  Galaxies, Ringberg Castle, Munich.  
\item [-] 2011, {\it Spectral and physical properties of high-z galaxies},
  Bridging Electromagnetic Astrophysics and Cosmology with Gravitational
  Waves, Milano.
\item [-] 2009 {\it Understanding High-z Lyman Alpha Emitters}, Fall
  Meeting German Astronomical Society, Potsdam 
\item [-] 2009 {\it The Milky Way in the Semi-Analytic Context}, Open
  Problems in Galaxy Formation, Potsdam 
%\item [-] 2008{\it Applications of the Semi-Analytic Approach to Galaxy Formation}, Seminar Cosmology Group, IA-UNAM, Mexico City
\item [-] 2008 {\it IR and submm galaxies in Semi-Analytic Models of
  Galaxy Formation}, Mexican Congress of Astronomy and Astrophysics,
  Mexico City 
%\item [-] 2006 {\it Luminous Infrared Galaxies: The Semi-Analytic Way}, Cosmology Group Seminar, Max Planck Institute for Astrophysics, Garching
%\item [-] 2006, {\it High-z galaxy surveys}, SKA-LOFAR PNC Workshop,
%Paris 
\item [-] Since 2006, different seminars at Max Planck Institute for
  Astrophysics, University of British Columbia, Marseille Observatory,
  Hebrew U. of Jerusalem, U. Nacional Aut\'onoma de M\'exico, U. de
  Antioquia (Colombia), Observatorio Astron\'omico Nacional
  (Colombia), U. Nacional de Colombia, U. de los Andes (Colombia), New
  Mexico State U., Arizona State U., Columbia U., Princeton U.,
  Harvard U., UC Berkeley, UC Santa Cruz, Pontificia Universidad
  Catolica (Chile), Universidad de Chile. 
\end{itemize}


{\bf Art \& Science}
\begin{itemize}
\item [-] 2013: Founding member of the project \emph{Astronom\'ia
  Perif\'erica} that aims at taking astronomy to Bogota's periphery
  through artistic interventions.  Secured funding for $\sim 5$KEuro
  from the Office of Astronomy for Development to kickstart the
  project. 
\item [-] 2012: Founding member and organizer of the Art and Science Meetings (Encuentros de Arte y Ciencia) in Bogot\'a.
\item [-] 2011: {\it Fluid Skies} secured funding for $\sim 50$ KEuro
  to make the Carved Air exhibition (Berlin, 2012), print the
  exhibition catalog and prepare academic meetings on the scientific
  and historical concepts of the project. The sponsors were: Ernst
  Schering Foundation (Germany), University of California Institute
  for Research in the Arts (USA) and the Arts Council Korea (ARKO).  
\item [-] Since 2011 member of the {\it Fluid Skies} collaboration
  together with Yunchul Kim (artist) and Lucia Ayala (Art
  Historian). {\it Fluid Skies} is a creative and research platform to
  explore the fluid materiality of the cosmos from the perspectives of
  astrophysics, art, history and philosophy. 
\item [-] 2010, Talk on {\it Simulations of Large Scale Structure
  Evolution}, Institut f\"ur Raumexperimente - Olafur Eliasson Studio,
  Berlin.  
\end{itemize}


{\bf Languages}
\begin{itemize}
\item[]Spanish (native), English (fluent), French (fluent), German (proficient)
\end{itemize}

%\vspace{1.0cm}
%\begin{flushright}
%{\it Last Updated: December/4/2011}
%\end{flushright}

\end{document}

