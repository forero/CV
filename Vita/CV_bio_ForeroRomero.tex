\documentclass[11pt]{article}
\newcommand{\galics}{{\sc GalICS}}
\newcommand{\momaf}{{\sc MoMaF}}
\newcommand{\lemomaf}{{\sc LeMoMaF}}
%\textheight=25.5cm
%\textwidth=16.0cm
%\oddsidemargin=-0.5cm
%\topmargin=-2.0cm
\pagenumbering{arabic}
\bibliographystyle{unsrt}

\begin{document}

\noindent

%==================================WHOAMI===============================================
%\colorbox{gris-claro}{{\Large Jaime Ernesto FORERO ROMERO}}\\
\noindent
{{\Large Jaime E. FORERO-ROMERO}}\\
%=================================NACIMIENTO============================================
\begin{flushleft}
Born October 14, 1981. Bogot\'a, Colombia\\
\end{flushleft}
%==================================DIRECCION============================================
\begin{flushright}
\verb"http://wwwprof.uniandes.edu.co/~je.forero/" 
\verb"http://github.com/forero"\\
\verb"je.forero at uniandes dot edu dot co"\\
Calle 18A\# 1 - 10\\
Bloque Ip, Of. 208\\
Universidad de los Andes\\
AA 4976, Bogot\'a, Colombia\\
\end{flushright}
\vspace{1cm}


{\bf Bio}
\begin{itemize}
\item[]Dr. Jaime E. Forero-Romero is a theoretical astrophysicist living in
Bogot\'a, Colombia. {\textit{Ancien \'el\`eve}} of the \'Ecole Normale
Sup\'erieure in Paris, after obtaining his PhD in 2007 he spent four
years in Berlin as a postdoctoral researcher in the Cosmology group at
the Leibniz-Institute for Astrophysics in Postdam. He was awarded
the Gruber Fellowship in 2011 by the International Astronomical Union
to spend one year in the Astronomy Department at the University of
California in Berkeley. During the same period he took part in different
Art \& Science projects, most notably taking part in the Fluid Skies
collaboration. Since mid 2012 he is an Assistant Professor in the
Physics Department at Universidad de los Andes in Bogot\'a. 
\end{itemize}

\vspace{1cm}


{\bf Education}
\indent
\begin{itemize}
\item[-] PhD. Physics, Ecole Normale Sup\'erieure de Lyon (France) 11/2007
\item[-] M.Sc.Physics (Magistère Interuniversitaire)  Ecole Normale
  Sup\'erieure (Paris, France), 08/2005
\item[-] Undergraduate Physics studies (3$^{rd}$-4$^{th}$ years),
  Instituto Balseiro (Argentina), 08/2001 - 08/2003
\item[-] Undergraduate Physics studies (1$^{st}$-2$^{nd}$ years),
  U. Nacional de Colombia, 07/1999 -  07/2001
\end{itemize}
\newpage
{\bf Academic Positions}
\begin{itemize}
\item[-] Assistant Professor, Physics Department, Universidad de los Andes (Colombia), 8/2012-
\item[-] Gruber Fellow at the Astronomy Department UC Berkeley (USA), 10/2011-7/2012
\item[-] Postdoctoral Researcher at the Leibniz Institute for
  Astrophysics-Potsdam (AIP) (Germany), 10/2007-9/2011  
\item[-] Graduate student at the Ecole Normale Sup\'erieure de Lyon
  (France), 09/2006-09/2007  
\item[-] \'El\`eve at the Ecole Normale Sup\'erieure (Paris, France),
  09/2003-08/2006 
\item[-] Undergraduate fellow at the Instituto Balseiro (Bariloche,
  Argentina), 08/2001-08/2003 
\end{itemize}


{\bf Honors and Awards}
\begin{itemize}
\item[-] 2011. IAU Peter and Patricia Gruber Foundation Fellowship
\item[-] 2006-2007. Graduate Research Fellowship (Allocation Coupl\'ee) of the French
  Ministry for Research and Education.
\item[-] 2003-2006. Scholarship at the Ecole Normale Sup\'erieure (Paris), covers tuition fees plus room and
  board to pursue graduate studies and PhD research.
\item[-] 2001-2003. Undergraduate scholarship (covers tuition fees at the
  Instituto Balseiro plus room and board) from the Argentinian Atomic Energy
  Comission.
\item[-] 1999. Gold Medal in the National Chemistry Olympiad, Bogot\'a,
  Colombia (First Place, National Competition)
\end{itemize}


{\bf Teaching}
\begin{itemize}
\item[-] 2013-II. {\it Computational Physics}. Universidad de los Andes.
\item[-] 2013-II. {\it Physics II (Thermodynamics and Electromagnetism)}. Universidad de los Andes.
\item[-] 2013-I. {\it Computational Physics}. Universidad de los Andes.
\item[-] 2013-I. {\it Physics I (Newtonian Mechanics)}. Universidad de los Andes.
\item[-] 2012-II. {\it Physics I (Newtonian Mechanics)}. Universidad de los Andes.
\item[-] 2012-II. {\it Introduction to Physics (module on Astrophysics)}. Universidad de los Andes.
\item[-]{\it Lectures on Galaxy Evolution}, V Colombian
  Astrophysics Summer School, Observatorio Astronomico Nacional, Summer 2009
\item[-]{\it{Introduction to Astrophysics (Teaching
      Assistant)}}, Undergraduate Level Course at the University of Lyon I,
      Spring 2007.
\end{itemize}



{\bf Publication Statistics}
\begin{itemize}
\item[]14 Refereed journal articles in astrophysics since 2007. Total
  Number of Citations: 176. H-index: 8. (Source: ADS)
\item[]2 Manuscripts in reviewing process.
\end{itemize}


{\bf Selected Talks}
\begin{itemize}
\item [] Since 2006, different seminars at Max Planck Institute for
  Astrophysics, University of British Columbia, Marseille Observatory,
  Hebrew U. of Jerusalem, U. Nacional Aut\'onoma de M\'exico, U. de
  Antioquia (Colombia), Observatorio Astron\'omico Nacional
  (Colombia), U. Nacional de Colombia, U. de los Andes (Colombia), New
  Mexico State U., Arizona State U., Columbia U., Princeton U.,
  Harvard U., UC Berkeley, UC Santa Cruz, Pontificia Universidad
  Catolica (Chile), Universidad de Chile. 
\end{itemize}


{\bf Art \& Science}
\begin{itemize}
\item [-] 2012: Founding member and organizer of the Art and Science Meetings (Encuentros de Arte y Ciencia) in Bogot\'a.
\item [-] Since 2011 member of the {\it Fluid Skies} collaboration
  together with Yunchul Kim (artist) and Lucia Ayala (Art
  Historian). {\it Fluid Skies} is a creative and research platform to
  explore the fluid materiality of the cosmos from the perspectives of
  astrophysics, art, history and philosophy. 
\item [-] 2010, Talk on {\it Simulations of Large Scale Structure
  Evolution}, Institut f\"ur Raumexperimente - Olafur Eliasson Studio,
  Berlin.  
\end{itemize}


{\bf Languages}
\begin{itemize}
\item[]Spanish (native), English (fluent), French (fluent), German (proficient)
\end{itemize}

%\vspace{1.0cm}
%\begin{flushright}
%{\it Last Updated: December/4/2011}
%\end{flushright}

\end{document}

