%%%%%%%%%%%%%%%%%%%%%%%%%%%%%%%%%%%%%%%%%
% Plasmati Graduate CV
% LaTeX Template
% Version 1.0 (24/3/13)
%
% This template has been downloaded from:
% http://www.LaTeXTemplates.com
%
% Original author:
% Alessandro Plasmati (alessandro.plasmati@gmail.com)
%
% License:
% CC BY-NC-SA 3.0 (http://creativecommons.org/licenses/by-nc-sa/3.0/)
%
%%%%%%%%%%%%%%%%%%%%%%%%%%%%%%%%%%%%%%%%%

%----------------------------------------------------------------------------------------
%	PACKAGES AND OTHER DOCUMENT CONFIGURATIONS
%----------------------------------------------------------------------------------------

\documentclass[a4paper,10pt]{article} % Default font size and paper size
\usepackage[utf8]{inputenc}
\usepackage[russian, english]{babel}

\usepackage{fontspec} % For loading fonts
\defaultfontfeatures{Mapping=tex-text}
%\setmainfont[SmallCapsFont = Fontin SmallCaps]{Fontin} % Main document font

\usepackage{xunicode,xltxtra,url,parskip} % Formatting packages

\usepackage[usenames,dvipsnames]{xcolor} % Required for specifying custom colors

\usepackage[big]{layaureo} % Margin formatting of the A4 page, an alternative to layaureo can be \usepackage{fullpage}
% To reduce the height of the top margin uncomment: \addtolength{\voffset}{-1.3cm}

\usepackage{hyperref} % Required for adding links	and customizing them
\definecolor{linkcolour}{rgb}{0,0.2,0.6} % Link color
\hypersetup{colorlinks,breaklinks,urlcolor=linkcolour,linkcolor=linkcolour} % Set link colors throughout the document

\usepackage{titlesec} % Used to customize the \section command
\titleformat{\section}{\Large\scshape\raggedright}{}{0em}{}[\color{black} \titlerule] % Text formatting of sections

\titleformat{\subsection}{\large\scshape\raggedright}{}{0em}{}[\color{black} \titlerule] % Text formatting of sections

\titleformat{\subsubsection}{\scshape\raggedright}{}{0em}{} % Text formatting of sections


\titlespacing{\section}{0pt}{4pt}{4pt} % Spacing around sections

\begin{document}

\pagestyle{empty} % Removes page numbering



%----------------------------------------------------------------------------------------
%	PERSONAL INFORMATION
%----------------------------------------------------------------------------------------

\par{\centering{\Huge \color{black}  Jaime E. FORERO-ROMERO} \bigskip\par} % Your name

\color{red}
\section{Personal Information}
\color{black}


\begin{tabular}{rl}
\textsc{Email:} & \href{mailto:je.forero@uniandes.edu.co}{je.forero@uniandes.edu.co}\\
\textsc{Website:} & \url{http://wwwprof.uniandes.edu.co/~je.forero}\\
\textsc{GitHub:} & \url{github.com/forero}\\
\textsc{Citizenship:} & Colombian\\
\textsc{Birth Date:} & October 14, 1981\\
\end{tabular}


%----------------------------------------------------------------------------------------
%	EDUCATION
%----------------------------------------------------------------------------------------

\color{red}
\section{Research Interests}
\color{black}
\begin{itemize}
\item{Cosmology and Galaxy Evolution}
\item{Computational Astrophysics}
\item{The Dark Energy Spectroscopic Instrument (Collaboration Member)}
\end{itemize}


\color{red}
\section{Education}
\color{black}


\begin{tabular}{rl}	
\textsc{November 2007} & {\bf PhD. Physics}\\
\textsc{August 2005}& at the Ecole Normale Sup\'erieure de Lyon (France) \\
&\\
\textsc{July 2005} & {\bf M.Sc.Physics (Magist\`ere Interuniversitaire)}\\  
\textsc{August 2003}& at the Ecole Normale Sup\'erieure (Paris, France) \\
&\\
\textsc{July 2003} & {\bf Undergraduate Physics studies} (3$^{rd}$ \& 4$^{th}$ year)\\
\textsc{August 2001} & at the Instituto Balseiro (Argentina)\\
&\\
\textsc{July 2001}  & {\bf Undergraduate Physics studies} (1$^{st}$ \& 2$^{nd}$ year)\\
\textsc{July 1999}& at the U. Nacional de Colombia\\
\end{tabular}

\color{red}
\section{Academic Positions}
\color{black}

\begin{tabular}{rl}	
 \textsc{Present Day} &  {\bf Associate Professor}\\
 \textsc{August 2015} & Physics Department, Universidad de los Andes (Colombia) \\
& \\
 \textsc{July 2015} &  {\bf Assistant Professor}\\
 \textsc{August 2012} & Physics Department, Universidad de los Andes (Colombia) \\
&\\
 \textsc{July 2012} &  {\bf Gruber Fellow}\\
 \textsc{October 2011} & Astronomy Department UC Berkeley (USA) \\
&\\
 \textsc{September 2011} &  {\bf Postdoctoral Researcher}\\
 \textsc{September 2007} & Leibniz Insititute for Astrophysics at Potsdam(Germany) \\ 
&\\
 \textsc{August 2007} &  {\bf Graduate Student}\\
 \textsc{August 2006} & Ecole Normale Sup\'erieure de Lyon (France)\\
&\\
 \textsc{July 2006} &  {\bf \'El\`eve}\\
 \textsc{September 2003} & Ecole Normale Sup\'erieure (Paris, France)\\
&\\
 \textsc{August 2003} &  {\bf Undergraduate Fellow}\\
 \textsc{August 2001} & Instituto Balseiro (Bariloche, Argentina)\\
\end{tabular}

\color{red}
\section{Research}
\color{black}

\subsection{Bibliometrics}
\begin{itemize}
\item Refereed journal articles:  26 published + 1 accepeted + 1 submitted (since
  first paper in 2007).
\item Google Scholar statistics. Total Number of Citations: 738. H-index: 14.
\item SAO/NASA Astrophysics Data System statistics. Total Number of Citations: 616. H-index: 13.
\end{itemize}
\subsection{Selected Publications}
\begin{itemize}


\item
{\it The Local Group in the cosmic web}, {\bf
  J.E. Forero-Romero}, R. Gonz\'alez, ApJ, 799, 1, 2015.

\item 
{\it Cosmic web alignments with the shape, angular momentum
  and peculiar velocities of dark matter halos}, 
  {\bf J.E. Forero-Romero}, S. Contreras, N. Padilla, MNRAS, 443,
  1090, 2014. 

\item {\it The kinematics of the Local Group in a cosmological context}, 
{\bf J.E. Forero-Romero}, Y. Hoffman, S. Bustamante, S. Gottloeber,
G. Yepes, ApJ Letters, 767, 1, 2013 

\item {\it Effects of Star Formation Stochasticity on the Ly
  $alpha$ \& Lyman Continuum Emission from Dwarf Galaxies}, {\bf
  J. E. Forero-Romero} \& M. Dijkstra, MNRAS 428, 2163, 2013 

\item 
{\it A Dynamical Classification of the  Cosmic Web}.  {\bf J.E. Forero-Romero} , Y. Hoffman ,  S. Gottloeber , A. Klypin, G. Yepes,
MNRAS, 396, 1815-1824, 2009
\end{itemize}

%\subsection{ArXiv-Only Publications}
\subsection{Supervised Postdocs}

\begin{itemize}
\item 2015-2016. Ver\'onica Arias. Dynamics of satellites in the
  Local Group.  
\end{itemize}

\subsection{Supervised Students}

\begin{itemize}
\subsubsection{2017}
\item[8] Sebasti\'an Franco Ulloa. 
  Undegraduate thesis in Physics at Universidad de los Andes. 
\emph{Lattice Boltzmann methods to simulate dark matter.}

\subsubsection{2016}
\item[7]
  Nicol\'as Romero D\'iaz.
  Undegraduate thesis in Physics at Universidad de los Andes. 
  \emph{Observational evidence of star formation stochasticity in the
    CALIFA dataset.}
\item[6] 
  David Esteban  Bernal Neira.
  Undegraduate thesis in Physics at Universidad de los Andes. 
  \emph{Acotando las velocidades tangenciales de las galaxias
    sat\'elite de Andr\'omeda utilizando optimizaci\'on no lineal.} 
\item[5] Sergio Hern\'andez Charpak. Undegraduate
  thesis in Physics at Universidad de los Andes. \emph{Laniakea in a cosmological context}. 
\subsubsection{2015}
\item[4] Mar\'ia Camila Remolina Guti\'errez. Undegraduate
  thesis in Physics at Universidad de los Andes. \emph{The joint
    effect of outflows and rotation on the Lyman-alpha line}. 
\item[3] Juan Nicol\'as Garavito Camargo. Master
  thesis in Physics at Universidad de los Andes. \emph{The effect of
    gas bulk rotation on the Lyman-alpha line}.
\item[2] Christian Nicanor Poveda Ruiz. Undegraduate thesis
  in Physics at Universidad de los Andes. \emph{A semi-analytic
    approach to formation processes in galaxies}. 
\subsubsection{2014}
\item[1] Camilo Andr\'es Rivera Lozano. Undergraduate thesis
  in Physics at Universidad de los Andes. \emph{Impacto de los
    par\'ametros cosmol\'ogicos en la 
    estructura a gran escala del Universo}.
\end{itemize}

\subsection{Grants}
\begin{itemize}
\item 2017-2021 (coPI) LACEGAL Latin American Chinese European GALaxy Formation network.
Funded by MSCA-RISE - Marie Sklodowska-Curie Research and Innovation Staff Exchange (RISE). (1.5 MEuro)
\item 2016-2020 (PI) COLCIENCIAS grant to work on the Dark
  Energy Spectroscopic Instrument. (60 KUSD)
\item 2012-2016 (PI) UNIANDES grant given as a startup package to
assistant professors (25 KUSD).
\item 2011-2012 (PI) IAU Peter and Patricia Gruber Foundation
  Fellowship. (50K USD).
\end{itemize}

%\subsection{Talks}

%\color{red}
%\section{Teaching}
%\color{black}
%\subsection{Lectures at Los Andes University}
%\subsection{Summer/Winter Schools}



\color{red}
\section{Service}
\color{black}

\subsection{International Organizing Committees}
\begin{tabular}{rl}	
 \textsc{Present Day} &  Coordinator for the Andean Regional Office of  Astronomy for Development.\\
 \textsc{October 2015} & \url{http://andean.astro4dev.org/}\\
& \\
 \textsc{October 2015} &  Main organizer of the Second
 \textit{Workshop Astronom\'ia en los Andes} \\
   & to convene astronomers in the Andean region.  Held in Bogot\'a, Colombia.\\
& \\
 \textsc{July 2013} &  Main organizer of the First
 \textit{Workshop Astronom\'ia en los Andes} \\
   & to convene astronomers in the Andean region.  Held in Bogot\'a, Colombia.\\
& \\
 \textsc{September 2011} &  Co-organizer of the international
 conference \textit{Young and Bright:}\\
 & \textit{understanding high redshigt structures}. Held in
 Potsdam, Germany.\\ 
\end{tabular}


%\subsection{National Organizing Committees}
%\subsection{International Refereeing}
%\subsection{National Refereeing}
%\subsection{Jury of Master/PhD Thesis}
%\subsection{Public Talks}



\color{red}
\section{Miscellanea}
\color{black}
%\subsection{Art \& Science}


%\begin{itemize}
%\subsubsection{2015}
%\item Collaboration with visual artist M\'onica Naranjo to
%  produce a \emph{Cosmological Calendar}. 
%\subsubsection{2014}
%\item Collaboration with architect and plastic artist Carlos
%  Moncada on the project \emph{Paisajes Estereoplanetarios} as part of
%  the \emph{Astronom\'ia Perif\'erica}. 
%\subsubsection{2013}
%\item Founding member of the project \emph{Astronom\'ia
%  Perif\'erica} that aims at taking astronomy to Bogota's periphery
%  through artistic interventions.  Secured funding for $\sim 5$KEuro
%  from the Office of Astronomy for Development to kickstart the
%  project.  
%\subsubsection{2012}
%\item Founding member and organizer of the Art and Science
%  Meetings (Encuentros de Arte y Ciencia) in Bogot\'a.  
%\subsubsection{2011}
%\item {\it Fluid Skies} secured funding for $50$ KEuro
%  to produce the Carved Air exhibition (Berlin, 2012), print the
%  exhibition catalog and prepare academic meetings on the scientific
%  and historical concepts of the project. The sponsors were: Ernst
%  Schering Foundation (Germany), University of California Institute
%  for Research in the Arts (USA) and the Arts Council Korea (ARKO).  
%\item Founding member of the {\it Fluid Skies} collaboration
%  together with Yunchul Kim (artist) and Lucia Ayala (Art
%  Historian). {\it Fluid Skies} is a creative and research platform to
%  explore the fluid materiality of the cosmos from the perspectives of
%  astrophysics, art, history and philosophy. 
%\subsubsection{2010}
%\item Talk on {\it Simulations of Large Scale Structure
%  Evolution}, Institut f\"ur Raumexperimente - Olafur Eliasson Studio,
%  Berlin.  
%\end{itemize}


\subsection{Languages}

\begin{tabular}{rl}
\textsc{Spanish:} & Native\\
\textsc{English:} & Fluent\\
\textsc{French:} & Fluent \\
\textsc{German:} & Proficient \\
\textsc{Russian:} & Beginner \\
\end{tabular}

\color{red}
\section{A Scientist Multiple Dimensions}
\color{black}

I build my practice as an astrophysist around broader
aspects of being embodied in this world in the current times. 
This is a constant play between seeding new relationships and
institutions; defining new objects of scientific scrutiny; exploring
aestethic dimensions; taking a critical stance of the role of the
scientist in semi-industrialized societies; taking care of the other.  


\subsection{Our place in the Universe}

The main driver of my research has been finding our place
in the Universe. By \emph{us} I mean our Galaxy, the Milky Way; by
\emph{Universe} I mean its material distribution on its larges scales.
describing the material
structure of our Universe on its largest scales. 
This had led me to work both in sumlated universes and most recently
on observational projects that aim at making maps of the universe.



Meeting.
Contradictions.
The first COCOA to the present.
The work in casa-B. Very local involvement with global connections and
cosmological feelings.
Caring.



Open-systems, critical engineering culture. All the power to the
user. 

\subsection{Universe Aestethics}

You wouldn't destroy a mountain for profits if you knew that
it took billions of years to be built. That's at least what I think. A
thought motivated by the experience of living in a country that
savages its territory looking for minerals to sell.

That's been my motivation to find out how can you \emph{feel} this knowldge,
which by definition means developing an \emph{aesthetic}, a feeling,
for the spatial, temporal and energetic scales that made it possible
to be here, right now, as human being reading this words, breathing
air. 

This has taken me to collaborate with artists, art historians,
humanists and many people beyond disciplines.
This is a quest to \emph{feel the Universe}.

\subsection{Astronomy for a Better World}

With the support of International Astronomical Union (IAU)
in 2015 we started a Regional Office of Astronomy for
Development (OAD) for the Andean Region \footnote{\url{http://andean.astro4dev.org/}}. 
\begin{quote}
The mission of the OAD is to help further the use of astronomy as a
tool for development by mobilizing the human and financial resources
necessary in order to realize the field’s scientific, technological
and cultural benefits to society.\footnote{\url{http://www.astro4dev.org/aboutiauoad/}} 
\end{quote}
We follow the structure of the central OAD in South Africa and focus
our work on the lines of research, education and outreach. 
The work we are doing is starting to yield results. 
This has included the realization that such undertaking is impossible
without the from the Social Science and the Humanities.

In my experience two outcomes have started to materialized.
First, challenging traditional and mercantilized concepts of
\emph{development} and second, toning down the mentality of efficiency and
success indicators to start building networks of \emph{mutual care}.  

\end{document}



\end{document}

\begin{tabular}{rl}	
\textsc{December 2016} & \textbf{Physics \& Systems and Computing Engineering Double Major} \\ 
\textsc{August 2012} & at Universidad de los Andes in Bogot\'a, Colombia. GPA: 4.45/5.0.  \\ 
 & \url{fisica.uniandes.edu.co/} \& \url{sistemas.uniandes.edu.co/} \\
&\\

\textsc{December} 2011 & \textbf{Bachelor} from Fundaci\'on Colegio UIS in Bucaramanga, Santander, \\
\textsc{January} 2001 & Colombia. \url{fcuis.edu.co/}\\

\end{tabular}


%----------------------------------------------------------------------------------------
%	RESEARCH
%----------------------------------------------------------------------------------------

\color{OrangeRed}
\section{Research Experience}
\color{black}

All the projects listed below have a wider explanation at my website under the following url: \url{mariacamilaremolinagutierrez.github.io/research}\\

\begin{tabular}{rl}
\textsc{Present Day} & \textbf{Summer Undergraduate Research Fellowship at NASA's Jet} \\
\textsc{June 2016} & \textbf{Propulsion Laboratory}. Title: Analysis of the size-flux relationship \\
& of the Point-Spread Function of near infrared detectors for gravitational \\ 
& weak lensing measurements. Advisors: Dr. Jason Rhodes, Dr. Andr\'es \\
& Plazas and Dr. Charles Shapiro. \\
&\\

\textsc{May 2016} & \textbf{Undergraduate physics thesis}. Title: Study of the influence of \\
\textsc{February 2014} & galaxy rotation and outflows in the Lyman Alpha spectral line. Advisor: \\
& Dr. Jaime Forero-Romero.\\
&\\

\textsc{December 2015} & \textbf{Research Assistant for a Ph.D. thesis}. Title: Astronomical image \\ 
\textsc{August 2015} & processing from large all-sky photometric surveys for the detection and \\
& the measurement of transients. Ph.D. Candidate: Juan Pablo Reyes. \\
& Advisor: Dr. Marcela Hern\'andez Hoyos. \\
&\\

\textsc{August} 2015 & \textbf{Summer Astrophysics ESA Program at Leiden Observatory} \\ 
\textsc{June} 2015 & Title: Ballet of the Free-Floating Planets. Advisor: Dr. Lucie J\'ilkov\'a. \\
\end{tabular}


%----------------------------------------------------------------------------------------
%	COMPUTER SKILLS 
%----------------------------------------------------------------------------------------

\color{OrangeRed}
\section{Computer Skills}
\color{black}
 
\begin{tabular}{rl}
Programming Languages: & Python (P), C (P), Java (P), Javascript (I), HTML (B).\\
Software: & GitHub, {\LaTeX}, \textsc{UNIX}.\\
Operating Systems: & Ubuntu, Windows, Mac.\\
&\\
Proficiency Levels: & P = Proficient, I = Intermediate, B = Basic.
\end{tabular}

% NEW PAGE
\newpage
% NEW PAGE

%----------------------------------------------------------------------------------------
%	SCHOOLS AND EVENTS
%----------------------------------------------------------------------------------------

\color{OrangeRed}
\section{Talks, Posters and Schools}
\color{black}

\begin{tabular}{rl}
\textsc{April} 2016 & Congress: Escape of Lyman radiation from galactic labyrinths  in Crete,\\ & Greece. Short talk.\\
&\\
\textsc{December} 2014 & Congress: IV Colombian Congress of Astronomy (COCOA) in Pasto,\\ & Colombia. Poster presentation.\\
&\\
\textsc{November} 2013 & Meeting: XIV Latin American Regional IAU Meeting (LARIM) in \\ & Florianopolis, Brazil. Poster presentation.\\
&\\
\textsc{July} 2013 & Workshop: Astronomía en los Andes (Astronomy in the Andes) in \\
& Bogota, Colombia. Short talk.\\
&\\
\textsc{June} 2011 & Summer Camp: Summer Science Program in Socorro, New Mexico,\\
& United States of America. \url{http://www.summerscience.org}\\
\end{tabular}


%----------------------------------------------------------------------------------------
%	SCHOLARSHIPS AND DISTINCTIONS
%----------------------------------------------------------------------------------------

\color{OrangeRed}
\section{Scholarships and Distinctions}
\color{black}

\begin{tabular}{rl}
\textsc{December} 2017 & \textbf{Quiero Estudiar Scholarship} for academic excellence that covers\\
\textsc{July} 2012&  tuition for undergraduate studies at Universidad de los Andes. \\ 
&\\
\textsc{November} 2011 & \textbf{Position 62} nationally in the colombian standarized tests (ICFES).\\ 
&\\
\textsc{November} 2011 & \textbf{Regional Andres Bello} distinction for the best mathematics score\\
& in Santander at the colombian standarized tests (ICFES).\\
&\\
\textsc{2007-2011} & Individual and regional representant at the \textbf{Colombian Mathematics}\\
& \textbf{Olympiads}. \url{http://oc.uan.edu.co/ocm/om98a.aspx} \\ 
&\\
\textsc{2006-2011} & \textbf{Honour Tution} for academic excellence in high school.\\
\end{tabular}


%----------------------------------------------------------------------------------------
%	SERVICE
%----------------------------------------------------------------------------------------

\color{OrangeRed}
\section{Science Volunteering}
\color{black}

\begin{tabular}{rl}

\textsc{July 2013} & Assistant in the creation of the \textbf{Astronomy Office for} \\
\textsc{January 2013} & \textbf{Development} for the Andean region.\\
&\\
\textsc{November} 2011 & Creation of RECA (Network of Colombian Astronomy Students).\\ 
& \url{https://sites.google.com/site/recastronomia/}\\
&\\
\textsc{2010-2011} & Director and editor of the scientific magazine \textbf{QUANTOS}.
\end{tabular}


%----------------------------------------------------------------------------------------
%	LANGUAGES
%----------------------------------------------------------------------------------------

\color{OrangeRed}
\section{Languages}
\color{black}

\begin{tabular}{rl}
\textsc{Spanish:} & Mother tongue.\\

\textsc{English:} & Fluent. (TOEFL iBT Score: 101/120)\\

\textsc{German:} & Learning. \\
\end{tabular}


%----------------------------------------------------------------------------------------
%	EXTRACURRICULAR ACTIVITIES
%----------------------------------------------------------------------------------------

\color{OrangeRed} 
\section{Extracurricular Activities}
\color{black}

\begin{itemize}
    \item Environmental Radio Broadcasting.
    \item Choir singing. Bongo drum.
    \item Squash, Volleyball, Swimming.
\end{itemize}

%----------------------------------------------------------------------------------------

\end{document}

%Lastly modified: July the 27th, 2016
