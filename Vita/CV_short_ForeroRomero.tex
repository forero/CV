%%%%%%%%%%%%%%%%%%%%%%%%%%%%%%%%%%%%%%%%%
% Plasmati Graduate CV
% LaTeX Template
% Version 1.0 (24/3/13)
%
% This template has been downloaded from:
% http://www.LaTeXTemplates.com
%
% Original author:
% Alessandro Plasmati (alessandro.plasmati@gmail.com)
%
% License:
% CC BY-NC-SA 3.0 (http://creativecommons.org/licenses/by-nc-sa/3.0/)
%
%%%%%%%%%%%%%%%%%%%%%%%%%%%%%%%%%%%%%%%%%

%----------------------------------------------------------------------------------------
%	PACKAGES AND OTHER DOCUMENT CONFIGURATIONS
%----------------------------------------------------------------------------------------

\documentclass[a4paper,10pt]{article} % Default font size and paper size

\usepackage{fontspec} % For loading fonts
\defaultfontfeatures{Mapping=tex-text}
%\setmainfont[SmallCapsFont = Fontin SmallCaps]{Fontin} % Main document font

\usepackage{xunicode,xltxtra,url,parskip} % Formatting packages

\usepackage[usenames,dvipsnames]{xcolor} % Required for specifying custom colors

\usepackage[big]{layaureo} % Margin formatting of the A4 page, an alternative to layaureo can be \usepackage{fullpage}
% To reduce the height of the top margin uncomment: \addtolength{\voffset}{-1.3cm}

\usepackage{hyperref} % Required for adding links	and customizing them
\definecolor{linkcolour}{rgb}{0,0.2,0.6} % Link color
\hypersetup{colorlinks,breaklinks,urlcolor=linkcolour,linkcolor=linkcolour} % Set link colors throughout the document

\usepackage{titlesec} % Used to customize the \section command
\titleformat{\section}{\Large\scshape\raggedright}{}{0em}{}[\color{black} \titlerule] % Text formatting of sections

\titleformat{\subsection}{\large\scshape\raggedright}{}{0em}{}[\color{black} \titlerule] % Text formatting of sections

\titleformat{\subsubsection}{\scshape\raggedright}{}{0em}{} % Text formatting of sections


\titlespacing{\section}{0pt}{4pt}{4pt} % Spacing around sections



\usepackage{fancyhdr}
\usepackage{lastpage}
 
\pagestyle{fancy}
\fancyhf{}
 
\rfoot{Page \thepage \hspace{1pt} of \pageref{LastPage}}

\begin{document}

%\pagestyle{empty} % Removes page numbering



%----------------------------------------------------------------------------------------
%	PERSONAL INFORMATION
%----------------------------------------------------------------------------------------

\par{\centering{\Huge \color{black}  Jaime E. FORERO-ROMERO} \bigskip\par} % Your name

\color{red}
\section{Personal Information}
\color{black}


\begin{tabular}{rl}
\textsc{Email:} & \href{mailto:je.forero@uniandes.edu.co}{je.forero@uniandes.edu.co}\\
\textsc{Personal Website:} & \url{http://wwwprof.uniandes.edu.co/~je.forero}\\
\textsc{Group Website:} & \url{http://astroandes.github.io/}\\ 
\textsc{Citizenship:} & Colombian\\
\textsc{Birth Date:} & October 14, 1981\\
\end{tabular}


%----------------------------------------------------------------------------------------
%	EDUCATION
%----------------------------------------------------------------------------------------
\color{red}
\section{Research Interests}
\color{black}
\begin{itemize}
\item{Cosmology} 
\item{Galaxy Formation}
\item{Computational Astrophysics}
\end{itemize}


\color{red}
\section{Education}
\color{black}


\begin{tabular}{rl}	
\textsc{November 2007} & {\bf PhD. Physics}\\
\textsc{August 2005}& at the \'Ecole Normale Sup\'erieure de Lyon (France) \\
&\\
\textsc{July 2005} & {\bf M.Sc.Physics (Magist\`ere Interuniversitaire)}\\  
\textsc{August 2003}& at the \'Ecole Normale Sup\'erieure (Paris, France) \\
&\\
\textsc{July 2003} & {\bf Undergraduate Physics studies} (3$^{rd}$ \& 4$^{th}$ year)\\
\textsc{August 2001} & at the Instituto Balseiro (Argentina)\\
&\\
\textsc{July 2001}  & {\bf Undergraduate Physics studies} (1$^{st}$ \& 2$^{nd}$ year)\\
\textsc{July 1999}& at the U. Nacional de Colombia\\
\end{tabular}

\color{red}
\section{Academic Positions}
\color{black}

\begin{tabular}{rl}	
 \textsc{Present Day} &  {\bf Associate Professor}\\
 \textsc{August 2015} & Physics Department, Universidad de los Andes (Colombia) \\
& \\
 \textsc{July 2015} &  {\bf Assistant Professor}\\
 \textsc{August 2012} & Physics Department, Universidad de los Andes (Colombia) \\
&\\
 \textsc{July 2012} &  {\bf Gruber Fellow}\\
 \textsc{October 2011} & Astronomy Department UC Berkeley (USA) \\
&\\
 \textsc{September 2011} &  {\bf Postdoctoral Researcher}\\
 \textsc{September 2007} & Leibniz Insititute for Astrophysics at Potsdam (Germany) \\ 
&\\
 \textsc{August 2007} &  {\bf Graduate Student}\\
 \textsc{August 2006} & \'Ecole Normale Sup\'erieure de Lyon (France)\\
&\\
 \textsc{July 2006} &  {\bf \'El\`eve}\\
 \textsc{September 2003} & \'Ecole Normale Sup\'erieure (Paris, France)\\
&\\
 \textsc{August 2003} &  {\bf Undergraduate Fellow}\\
 \textsc{August 2001} & Instituto Balseiro (Bariloche, Argentina)\\
\end{tabular}


\newpage

\color{red}
\section{Research}
\color{black}

\subsection{Bibliometrics}
\begin{itemize}
\item Refereed journal articles:  44 articles (11.2, normalized count) since first paper in 2007. 
\item Google Scholar statistics. Total Number of Citations: 1695. H-index: 17.
\item SAO/NASA Astrophysics Data System statistics. Total Number of
  Citations: 1101. H-index: 15. 
\end{itemize}


\subsection{Selected Publications}
\begin{itemize}


 \item {\it Classifying image sequences of astronomical transients
   with deep neural networks}, 
   C. G\'omez, M. Neira,  M. Hernández Hoyos, P. Arbeláez,
   {\bf J. E. Forero-Romero}, MNRAS, 499, 3, 2020.
   
\item {\it Dark matter halo shapes in the Auriga simulations}, 
  J. Prada, {\bf J. E. Forero-Romero}, R. J. J. Grand, R. Pakmor, 
  V. Springel,  MNRAS, 490, 4877, 2019.

\item{\it The Local Group in the cosmic web}, {\bf
  J.E. Forero-Romero}, R. Gonz\'alez, ApJ, 799, 1, 2015.

\item{\it Cosmic web alignments with the shape, angular momentum
  and peculiar velocities of dark matter halos}, 
  {\bf J.E. Forero-Romero}, S. Contreras, N. Padilla, MNRAS, 443,
  1090, 2014. 

\item  {\it A Dynamical Classification of the  Cosmic Web}.  {\bf
  J. E. Forero-Romero} , Y. Hoffman,  S. Gottloeber, A. Klypin, G. Yepes, 
MNRAS, 396, 1815-1824, 2009   

\end{itemize}

\subsection{Grants}
\begin{itemize}
\item 2017-2021 (coPI) LACEGAL Latin American Chinese European GALaxy Formation network.
Funded by MSCA-RISE - Marie Sklodowska-Curie Research and Innovation Staff Exchange (RISE). (1.5 MEuro)
\item 2016-2019 (PI) COLCIENCIAS grant to work on the Dark
  Energy Spectroscopic Instrument. (60 kUSD)
\item 2012-2016 (PI) UNIANDES grant given as a startup package to
assistant professors (25 kUSD).
\item 2011-2012 (PI) IAU Peter and Patricia Gruber Foundation
  Fellowship. (50 kUSD).
\end{itemize}

\color{red}
\section{Mentoring}
\color{black}

%\subsection{ArXiv-Only Publications}
\subsection{Postdocs}

\begin{itemize}
\item 2019-. David Sierra. Building the Bright Galaxy Sample catalog 
with data from the Dark Energy Spectroscopic Instrument.
\item 2015-2016. Ver\'onica Arias. Dynamics of satellites in the
  Local Group.  
\end{itemize}

\subsection{Graduate Students}

\begin{itemize}

\item 2019- John Suarez. PhD thesis on cosmic web reconstruction.
\item 2018- Yeimy Camargo. PhD thesis on galaxy assembly bias.
\item 2019 Felipe G\'omez. Master thesis in Physics.
\emph{A large scale structure void finder based on the $\beta$-skeleton graph.}
\item 2016-2018 Jes\'us Prada.
  Master thesis in Physics.
\emph{Constraining the dark matter halo shape of the Milky Way with
  numerical simulations.} 
\item 2013-2015 Juan Nicol\'as Garavito Camargo. Master
  thesis in Physics. \emph{The effect of
    gas bulk rotation on the Lyman-alpha line}.
\end{itemize}



%\subsection{Talks}

%\color{red}
%\section{Teaching}
%\color{black}
%\subsection{Lectures at Los Andes University}
%\subsection{Summer/Winter Schools}


\color{red}
\section{Service}
\color{black}
\subsection{Scientific Collaborations}

\begin{tabular}{rl}	
 \textsc{Present Day} &  Full Member of the Dark Energy Spectroscopic Instrument.\\
 \textsc{January 2014} & \url{http://desi.lbl.gov/}\\
& \\ 
\textsc{Present Day} & Referee for The Astrophysical Journal and \\
\textsc{January 2010} & Monthly Notices of the Royal Astronomical
Society\\
& \\ 
\textsc{May 2021} & External Reviewer for the Hubble Space Telescope \\%cycle
%28 (2020), cycle 29 (2021)
\textsc{May 2020} &\\
& \\ % observing program\\
  \textsc{October 2020} &  Coordinator for the Andean Regional Office of  Astronomy for Development.\\
 \textsc{October 2015} & \url{http://andean.astro4dev.org/}\\
& \\

\end{tabular}

\subsection{Event Organizer}
\begin{tabular}{rl}
 \textsc{October 2015} &  Main organizer of the Second
 \textit{Workshop Astronom\'ia en los Andes} \\
   & to convene astronomers in the Andean region.  Held in Bogot\'a, Colombia.\\
& \\
 \textsc{June 2015} &  Main organizer for the 
 \textit{Andean School on Cosmology} \\
   & held in Bogot\'a, Colombia.\\
& \\
 \textsc{July 2013} &  Main organizer of the First
 \textit{Workshop Astronom\'ia en los Andes} \\
   & to convene astronomers in the Andean region.  Held in Bogot\'a, Colombia.\\
& \\
 \textsc{September 2011} &  Co-organizer of the international
 conference \textit{Young and Bright:}\\
 & \textit{understanding high redshigt structures}. Held in
 Potsdam, Germany.\\ 
& \\

\end{tabular}


\subsection{Guest Lecturer}
\begin{tabular}{rl}
 \textsc{July 2018} &  Lecturer for the
 \textit{41st International School for Young Astronomers} \\
   & held in Socorro, Colombia.\\
& \\
 \textsc{December 2015} &  Lecturer for the 
 \textit{Second Guatemalan School of Astrophysics} \\
   & held in Antigua, Guatemala.\\
& \\
 \textsc{June 2013} &  Lecturer for the 
 \textit{Pan-American Advanced Studies Institutes (PASI)} \\
   &  on Methods in Computation-Based Discovery.  Held in Guatemala City, Guatemala.\\
& \\
\textsc{July 2009} & Lecturer for the 
\textit{Fifth Colombian School of Astronomy}\\
& held in Bogot\'a, Colombia.\\

\end{tabular}

%\subsection{National Organizing Committees}
%\subsection{International Refereeing}
%\subsection{National Refereeing}
%\subsection{Jury of Master/PhD Thesis}
%\subsection{Public Talks}


\end{document}

