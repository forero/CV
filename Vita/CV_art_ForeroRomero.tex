\documentclass[letterpaper,11pt,onecolumn]{article}
\usepackage[spanish]{babel}
\usepackage[latin1]{inputenc}
\usepackage[pdftex]{color,graphicx}
\usepackage{hyperref}
\setlength{\oddsidemargin}{0cm}
\setlength{\textwidth}{490pt}
\setlength{\topmargin}{-40pt}
\addtolength{\hoffset}{-0.3cm}
\addtolength{\textheight}{4cm}

%\documentclass[9pt]{article}
%\newcommand{\galics}{{\sc GalICS}}
%\newcommand{\momaf}{{\sc MoMaF}}
%\newcommand{\lemomaf}{{\sc LeMoMaF}}
%\textheight=25.5cm
%\textwidth=16.0cm
%\oddsidemargin=-0.5cm
%\topmargin=-2.0cm
\pagenumbering{arabic}
\bibliographystyle{unsrt}

\begin{document}

\noindent

%==================================WHOAMI===============================================
%\colorbox{gris-claro}{{\Large Jaime Ernesto FORERO ROMERO}}\\
\noindent
{{\Large Jaime E. FORERO-ROMERO}}\\
%=================================NACIMIENTO============================================
\begin{flushleft}
Born October 14, 1981. Bogot\'a, Colombia\\
\end{flushleft}
%==================================DIRECCION============================================
\begin{flushright}
\url{http://wwwprof.uniandes.edu.co/~je.forero/} \\{\tt{je.forero at uniandes dot edu dot co}}\\
\url{https://github.com/forero}\\
Calle 18A\# 1 - 10\\
Bloque Ip, Of. 208\\
Universidad de los Andes\\
AA 4976, Bogot\'a, Colombia\\
\end{flushright}


{\bf Main Scientific Research Interests}
\begin{itemize}
\item[-]{Computational Cosmology and Galaxy Formation}
\item[-]{Large Scale Structure of the Universe}
\item[-]{The Dark Energy Spectroscopic Instrument (DESI)}
\end{itemize}

{\bf Main Artistic Research Interests}
\begin{itemize}
\item[-]{Materiality of the Cosmos}
\item[-]{Personal Cosmologies}
\item[-]{Reality as a Pluriverse}
\item[-]{Utopian Communities}
\end{itemize}

{\bf Education}
\indent
\begin{itemize}
\item[-] PhD. Physics, Ecole Normale Sup\'erieure de Lyon (France) 11/2007
\item[-] M.Sc.Physics (Magist\`ere Interuniversitaire)  Ecole Normale
  Sup\'erieure (Paris, France), 08/2005
\item[-] Undergraduate Physics studies (3$^{rd}$ \& 4$^{th}$ year),
  Instituto Balseiro (Argentina), 08/2001 - 08/2003
\item[-] Undergraduate Physics studies (1$^{st}$ \& 2$^{nd}$ year),
  U. Nacional de Colombia, 07/1999 -  07/2001
\end{itemize}

{\bf Academic Positions}
\begin{itemize}
\item[-] Associate Professor, Physics Department, Universidad de los Andes (Colombia), 8/2015-
\item[-] Assistant Professor, Physics Department, Universidad de los Andes (Colombia), 8/2012-7/2015
\item[-] Gruber Fellow at the Astronomy Department UC Berkeley (USA), 10/2011-7/2012
\item[-] Postdoctoral Researcher at the Leibniz Institute for
  Astrophysics-Potsdam (AIP) (Germany), 10/2007-9/2011  
\item[-] Graduate student at the Ecole Normale Sup\'erieure de Lyon
  (France), 09/2006-09/2007  
\item[-] \'El\`eve at the Ecole Normale Sup\'erieure (Paris, France),
  09/2003-08/2006 
\item[-] Undergraduate fellow at the Instituto Balseiro (Bariloche,
  Argentina), 08/2001-08/2003 
\end{itemize}

\newpage




{\bf Collaborations}
\begin{itemize}
\item [-] 2016: {\bf The Language that Cosmology
  Speaks}. Collaboration with Diana Murcia (designer). We are
  designing a series of posters on the common thread tying living
  cosmologies (traditional, contemporary, physical, metaphysical). 
\item [-] 2015: {\bf Cosmic Calendar 2016}. Collaboration with M\'onica
  Naranjo (visual artist). We designed and produced a 2016 calendar
  depicting the cosmological evolution of the Universe in the standard
  Big Bang cosmology. 
\item [-] 2015: {\bf Biodharma Theater Summer School}. Collaboration
  with Beatriz Camargo (drama- / dharma- turgist). We gave three lectures and
  practical exercises on contemporaty physical cosmology for theater students.  
\item [-] 2014: {\bf Stereoplanetary Landscapes}. Collaboration with
  Carlos Moncada (designer), Dario Sendoya (sociologist/activist) and
  Nicol\'as Garavito (scientist).
\item [-] 2013: {\bf Peripheral Astronomy}. Collaboration with Luis
  Bustamante (designer/media artist), Alejandro Tamayo (artist) and Manuel
  Santana (plastic artist), Dario Sendoya (sociologist/activist) and
  Johanna Villamil (designer/artist).
  Founding member of the project \emph{Astronom\'ia
  Perif\'erica} that aims at taking astronomy to Bogota's periphery
  through artistic interventions.  Secured funding for $\sim 5$KEuro
  from the Office of Astronomy for Development to kickstart the
  project. 
\item [-] 2012: {\bf Bogot\'a Art and Science Meetings}. Collaboration
  with Alejandro Tamayo (artist) and Johanna Villamil
  (designer/artist). 
\item [-] 2011-2012: {\bf Fluid Skies}. Collaboration with Yunchul Kim
  (artist) and Luc\'ia Ayala (art historian). 
{\it Fluid Skies} is a creative and research platform to
  explore the fluid materiality of the cosmos from the perspectives of
  astrophysics, art, history and philosophy. 
\end{itemize}

{\bf Grants}
\begin{itemize}
\item[-] 2014. (CoPI) Stereoplanetary Landscapes
\item[-] 2013. (PI) Astronomy for Development Grant for the art/science collaborative project \emph{Astronom\'ia Perif\'erica} (5K USD). 
\item [-] 2011: {\it Fluid Skies} secured funding for $\sim 10$K USD
  from the University of California Institute for Research in the Arts
  (USA) to support the Carved Air exhibition (Berlin, 2012).
\end{itemize}


{\bf Languages}
\begin{itemize}
\item[]Spanish (native), English (fluent), French (fluent), German (proficient)
\end{itemize}

\end{document}

