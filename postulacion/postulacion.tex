\documentclass{book}
%\documentclass[12pt,letterpaper]{article}
%\documentclass[12pt]{article}                                                                                     
\usepackage{natbib}
\usepackage{tabularx}
\usepackage[spanish]{babel}
\usepackage[latin1]{inputenc}
\usepackage[pdftex]{color,graphicx}
\usepackage[absolute]{textpos} % Para poner una imagen completa en la portada                                      
\usepackage{bookman}
\usepackage{hyperref}
\usepackage[T1]{fontenc}
%\usepackage{draftwatermark}                                                                                       
%\SetWatermarkText{DRAFT}                                                                                          
%\SetWatermarkScale{1}                                                                                             
%\setlength{\oddsidemargin}{0cm}
%\setlength{\textwidth}{450pt}
%\pagenumbering{arabic}


\title{Postulaci\'on al premio academia mundial de ciencias (TWAS)
  para j\'ovenes cient\'ificos colombianos} 
\date{30 de Abril del 2018}
\author{Jaime E. Forero-Romero\\Universidad de los Andes\\Bogot\'a,
  Colombia} 
\begin{document}
\maketitle


%\begin{textblock*}{100mm}(82mm,20mm)
\begin{flushleft}
Academia Colombiana de Ciencias Exactas, F\'isicas y Naturales\\
Carrera 28 A No. 39 A 63\\
Bogot\'a D. C.\\
Colombia\\
\end{flushleft}
\begin{flushright}
\small
\noindent
  Jaime E. Forero-Romero\\
  Profesor Asociado\\
  Calle 18A \# 1 - 01\\
  Bloque Ip, Of 208\\
  Departamento de F\'isica\\
  Universidad de los Andes\\
  Bogot\'a, Colombia\\
\url{je.forero@uniandes.edu.co}\\
\url{http://wwwprof.uniandes.edu.co/~je.forero/}\\
\end{flushright}
%\end{textblock*}

\vspace*{20mm}

\noindent
Abril 30, 2018\\
{\bf Asunto: Presentaci\'on postulaci\'on candidatura premio TWAS para
j\'ovenes cient\'ificos columbianos.}\\

\noindent

Con esta carta presento los documentos necesarios para la
postulaci\'on al premio TWAS para j\'ovenes cient\'ificos colombianos.

Regres\'e a Colombia en Julio del 2012 para ser profesor de planta en
la Universidad de los Andes en Bogot\'a. 
Estuve por fuera de Colombia durante 11 a\~nos para completar mis
estudios de F\'isica y Astronom\'ia en Argentina y Francia, y luego en
estancias postdoctorales en Alemania y Estados Unidos.
En Colombia he tenido la fortuna de poder dirigir tesis de pregrado y
posgrado, adem\'as de apoyar la realizaci\'on de escuelas y de
congresos  para impulsar el desarrollo de la astronom\'ia en el pa\'is y la
regi\'on. Todo esto manteniendo mientras mantengo colaboraci\'ones
internacionales del m\'as alto nivel cient\'ifico. 

En las siguientes p\'aginas incluyo: una s\'intesis de mi obra
integral detallando mis aportes a la ciencias, una hoja de vida que
resume mi actividad investigativa, una copia de mi c\'edula de
ciudadan\'ia y copias de todas mis publicaciones.

Agradezco su tiempo al considerar esta postulaci\'on y quedo atento a
cualquier aclaraci\'on que se pueda necesitar.

Saludos cordiales,
\\
\\
\\
\\
\\
\\
\vspace{2cm}
\noindent
%\includegraphics[width=3.5cm]{jaime_firma.jpg}\\
Jaime E. Forero-Romero

\newpage

\subsection*{S\'intesis de aportes a la ciencia}

Desde hace 15 a\~nos mi labor investigativa se centra en la
formaci\'on de galaxias y cosmolog\'ia computacional. Obtuve mi
doctorado con estudios en esta \'area, mis estancias postdoctorales
fueron en institutos y universidades donde se investigaba en estos
temas, y desde hace 6 a\~nos coordino un equipo de estudiantes e
investigadores en \'areas de astrof\'isica computacional dentro del
grupo de Astrof\'isica de la Universidad de los Andes en Bogot\'a.  

El principal aporte de mis investigaciones se ha hecho en el \'area
del estudio de la estructura a gran escala del Universo. Mapas del
Universo que se realizan desde comienzos de la d\'ecada de los 80
revelan que las galaxias se distribuyen sobre grandes filamentos que
reciben el nombre de la \emph{red c\'osmica}. Quedaba abierta la
pregunta sobre la mejor manera de entrenar computadoras para que
encontraran esas estructuras.

Desde el 2009 hasta la actualidad mi l\'inea principal de
investigaci\'on, y la que m\'as atenci\'on ha recibido por parte de la
comunidad internacional, es la del {\bf desarrollo de algoritmos para que
un computador pueda encontrar y caracterizar la red c\'osmica}. 
Estos avances nos han permitido estudiar
c\'omo diferentes procesos de evoluci\'on de galaxias cambian de
acuerdo a su ubicaci\'on en la red c\'osmica. En particular, le he
prestado atenci\'on a c\'omo la red c\'osmica influye la evoluci\'on
de nuestra galaxia, la V\'ia L\'actea, para entender {\bf c\'omo nuestra
galaxia es at\'ipica una vez se la considera en cuenta un contexto
cosmol\'ogico amplio}. 
El desarrollo de las herramientas computacionales para el estudio de
la estructura del Universo a gran escala tambi\'en me han permitido
realizar aportes en otras \'areas como lentes gravitacionales y 
evoluci\'on de galaxias j\'ovenes. 

Actualmente a trav\'es de mi
gesti\'on y experiencia en cosmolog\'ia computacional la Universidad
de los Andes hace parte del Dark Energy Spectroscopic Instrument
(DESI), liderado por Berkeley Lab en Estados Unidos, el experimento
que har\'a desde el 2019 hasta el 2024 el mapa m\'as detallado del
Universo con el objetivo de entender la Energ\'ia Oscura. 
Es la primera vez que una instituci\'on colombiana hace parte de un
gran experimento de cosmolog\'ia observacional. 











\end{document}
